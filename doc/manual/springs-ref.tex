\section{{\tt springs}: separate universes for Lua} 

\subsection{Origins and purpose}
Springs (Serialization through Pluto for RINGS) is an extension of Lua Rings and
Pluto: Lua Rings allow to create new Lua states from within Lua, but offers
limited communication between them: a master universe can only send instruction
to a slave universe through a ``{\tt dostring}'', and the slave universe can
only send back strings, integers and booleans as results. Since Pluto allows to
serialize pretty much any Lua value as a string, it's used to create powerful
bidirectional communications between universes.

Springs is used internally by metalua to prevent different files' compile time
actions to interfere with each other: each file is compiled on a fresh clean
single-use slate.

The underlying projects can be found on the web:
\begin{itemize}
\item \verb|<http://www.keplerproject.org/rings/>|
\item \verb|<http://luaforge.net/projects/pluto/>|
\end{itemize}
Notice however that the Pluto version used in metalua has significantly patched
and debugged by Ivko Stanilov.

\subsection{API}
Go to Lua Rings web site for a reference on its original API. This API is
extended by spring with:
\begin{itemize}
\item function {\tt springs.new()} which creates a new universe ready for Pluto
  communication;
\item ({\tt:dostring()} works as usual)
\item {\tt :pcall(f, arg1, ..., argn)} works as standard function pcall(),
  except that execution occurs in the sub-state. Arguments are passed and
  results are returned transparently acrosse universes. Moreover, 'f' can also
  be a string, rather than a function. If it's a string, it must eval to a
  function in the substate's context. This allows to pass standard functions
  easily. For instance:\\
  \verb|r:pcall('table.concat', {'a', 'b', 'c'}, ',')|
\item {\tt :call()} is similar to :pcall(), except that in case of error, it
  actually throws the error in the sender universe's context. Therefore, it
  doesn't return a success status as does pcall(). For instance: \\
  \verb|assert('xxx' == r:call('string.rep', 'x', 3))|
\end{itemize}

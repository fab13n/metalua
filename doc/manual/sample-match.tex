\subsection{Structural pattern matching}

FIXME: refer to the official match extension instead of re-explaining
what pattern matching is about.

\subsubsection{Basic principles}
In many languages, including Lua, ``pattern matching'' generally
refers to the ability to:
\begin{itemize}
\item Analyse the general shape of a string;
\item capture specified parts of it into temporary variables
\item do some computation when the string matches a certain pattern,
  generally using the temporary captured variables.
\end{itemize}

In languages derived from ML\footnote{for instance OCaml, SML, Haskell
  or Erlang.}, pattern matching can be done on arbitrary data, not
only strings. One can write patterns which describe the shape of a
data structure, bind some interesting sub-parts of it to variables,
and execute some statements associated with a given pattern whenever
the data matches.

This sample aims to implement this capability into metalua. It will
discuss:
\begin{itemize}
\item How to describe data patterns (it turns out that we'll hijack
  metalua's expression parser, which is a great time-saving trick when
  it can be pulled).
\item How to compile them into equivalent code.
\item How to wrap all this up into a working compiled statement
\item What improvements can be done to the proposed design.
\end{itemize}

\subsubsection{Patterns}
A match statement consists of a tested term, and a series of (pattern,
block) pairs. At execution, the first pair whose pattern accurately
describes the tested term is selected, and its corresponding block is
evaluated. Other blocks are ignored. 

We will keep the implementation of patterns as simple as possible. To
do that, we will put a first constraint: the AST of a pattern must be
a legal expression AST. This way, we save most of the syntax handling
trouble. More specifically:

\begin{itemize}
\item numbers are valid patterns, which only match identical numbers;
\item strings are valid patterns, which only match identical strings;
\item tables are valid patterns if all of their keys are integers or
  strings, and all of their values are valid patterns. A table pattern
  matches a tested term if:
  \begin{itemize}
  \item the tested term is a table;
  \item every key that appears in the pattern also appears in the
    tested term;
  \item for every key, the value associated to this key in the tested
    term is matched by the sub-pattern associated to this key in the
    pattern;
  \end{itemize}
\item variables are valid patterns, and match everything. Moreover,
  the term matched by the variable captures it, i.e. in the
  corresponding block, that variable is set to the
  matched term.
\end{itemize}

Notice that since \verb|tag| is a regular field in metalua (albeit
with an optional dedicated syntax), it doesn't need a special case in
our pattern definition.

\paragraph{Example} Let's consider implementing an evaluator for
Metalua expressions. We will restrict ourselves to binary operators,
variables and numbers; we will also consider that we have a
\verb|values| table mapping variable names to their value, and
\verb|binopfuncs| which associate an operator name with the corresponding
function. The evaluator would be written:

\begin{Verbatim}[fontsize=\scriptsize]

function eval(t)
   match t with
   | `Op{ op, a, b } -> return binopfuncs[op](eval(a), eval(b))
   | `Number{ n }    -> return n
   | `Variable{ v }  -> return values[v]
   end
end
\end{Verbatim}

\subsubsection{Pattern compilation}

A pattern in a case will translate into:
\begin{itemize}
\item tests: testing that the type of the tested case is the same as
  the pattern's type for numbers, strings and tables;
\item local declarations: a variable must be bound to the
  tested term's value;
\item nested tests for every pattern key/value pair, when the pattern
  is a table. Moreover, since there might be multiple tests to run
  against the tested term's field, it should be put in a temporary
  variable.
\end{itemize}

For instance, consider the following pattern:

\begin{Verbatim}[fontsize=\scriptsize]

{ tag = "Foo", 10, { 20 }, { x = a }, b }
\end{Verbatim}

It corresponds to the series of tests and assignments on the left:

\begin{minipage}{6cm}
\begin{Verbatim}[fontsize=\scriptsize]

let v1 = <tested_term>
type(v1) == "table"
let v2 = v1.tag
v2 ~= nil
type(v2) == "string"
v2 == "Foo"
let v2 = v1[1]
v2 ~= nil
type(v2) == "number"
v2 == 10
let v2 = v1[2]
v2 ~= nil
type(v2) == "table"
let v3 = v2[1]
v3 ~= nil
type(v3) == "number"
v3 == 20
let v2 = v1[3]
v2 ~= nil
type(v2) == "table"
let v3 = v2.x
v3 ~= nil
local a = v3
let v2 = v1[4]
v2 ~= nil
local b = v2


\end{Verbatim}
\end{minipage}
\begin{minipage}{6cm}
\begin{Verbatim}[fontsize=\scriptsize]

let v1 = tested_term
if type(v1) == "table" then
 let v2 = v1.tag
 if v2 ~= nil then
  if type(v2) == "string" then
   if v2 == "Foo" then
    let v2 = v1[1]
    if v2 ~= nil then
     if type(v2) == "number" then
      if v2 == 10 then
       let v2 = v1[2]
       if v2 ~= nil then
        if type(v2) == "table" then
         let v3 = v2[1]
         if v3 ~= nil then
          if type(v3) == "number" then
           if v3 == 20 then
            let v2 = v1[3]
            if v2 ~= nil then
             if type(v2) == "table" then
              let v3 = v2.x
              if v3 ~= nil then
               local a = v3
               let v2 = v1[4]
               if v2 ~= nil then
                local b = v2
                <inner_term>
end ... end -- (16 times)
\end{Verbatim}
\end{minipage}
~\\~\\

Notice that the relative order of tests and assignments is meaningful:
we cannot put all assignments on one side, and all tests on an
other, e.g. \verb|v2 = v1.tag| on line 3 doesn't make sense if
\verb|type(v1) == table| on line 2 fails.

We will compile patterns in several steps: first, accumulate all the
tests and assignments in order; then, we will collapse them in a
single big nesting of ``if'' statements. At first, we won't try
to optimize the form of the final term. The list above left would be
collapsed into the single compound statement above on the right.

\paragraph{Accumulating constraints and tests}
This is done by a \verb|parse_pattern()| function, which does just what
is described in the bullet list above:

\begin{Verbatim}[fontsize=\scriptsize]

      -------------------------------------------------------------------
      -- Turn a pattern into a list of conditions and assignations,
      -- stored into [acc]. [n] is the depth of the subpattern into the
      -- toplevel pattern; [tested_term] is the AST of the term to be 
      -- tested; [pattern] is the AST of a pattern, or a subtree of that
      -- pattern when [n>0].
      -------------------------------------------------------------------
      local function parse_pattern (n, pattern)
         local v = var(n)
         if "Number" == pattern.tag or "String" == pattern.tag then
            accumulate (+{ -{v} == -{pattern} })
         elseif "Id" == pattern.tag then
            accumulate (+{stat: local -{pattern} = -{v} })
         elseif "Table" == pattern.tag then
            accumulate (+{ type( -{v} ) == "table" } )
            local idx = 1
            for _, x in ipairs (pattern) do
               local w = var(n+1)
               local key, sub_pattern
               if x.tag=="Key" 
               then key = x[1];           sub_pattern = x[2]
               else key = `Number{ idx }; sub_pattern = x; idx=idx+1 end
               accumulate (+{stat: (-{w}) = -{v} [-{key}] })
               accumulate (+{ -{w} ~= nil })
               parse_pattern (n+1, sub_pattern)
            end
         else error "Invalid pattern type" end
      end
      -------------------------------------------------------------------
\end{Verbatim}

This function relies on the following helper functions:
\begin{itemize}
\item {\tt var($n$)} generates the variable name ``\verb|v|$n$'', which
  is used to store the tested term at depth level $n$. Indeed,
  sub-patterns in table fields are matched against sub-terms of the
  tested term. It also remembers of the biggest $n$  it ever received, 
  and stores it into \verb|max_n| (this will be used to know which
  local vars have to be generated, see below);
\item {\tt accumulate()} just stores an additional code snippet in a
  dedicated list;
\end{itemize}

In the quotes, you might notice the parentheses around the variable in
``\verb|(-{w}) = -{v} [-{key}]|'': they're here to let the compiler
know that what's in \verb|-{...}| is an expression, not a
statement. Without them, it would expect {\tt w} to be the AST of a
statement (since we're in a statement context at this point), then
choke on the unexpected ``='' symbol. This is a common idiom, which
you should think about everytime you generate an assignment to an
anti-quoted identifier.

\paragraph{Collapsing the accumulated quotes}
As written above, our collapsing function will be kept as simple
as possible, and will not try to minimize the amount of generated
code. It takes as parameters {\tt n}, the index of the quote currently
collapsed in the accumulator, and {\tt inner\_term} the statement
block to put inside the innermost part of the test. It calls itself
recursively, so that the collapsed term is built inside out (generally
speaking, working with trees, including AST, involves a lot of
recursive functions). \verb|acc| is the list filled by the
\verb|accumulate()| function:

\begin{Verbatim}[fontsize=\scriptsize]

      -------------------------------------------------------------------
      -- Turn a list of tests and assignations into [acc] into a
      -- single term of nested conditionals and assignments.
      -- [inner_term] is the AST of a term to be put into the innermost
      -- conditionnal, after all assignments. [n] is the index in [acc]
      -- of the term currently parsed.
      -- 
      -- This is a recursive function, which builds the inner part of
      -- the statement first, then surrounds it with nested 
      -- [if ... then ... end], [local ... = ...] and [let ... = ...]
      -- statements.
      -------------------------------------------------------------------
      local function collapse (n, inner_term)
         assert (not inner_term.tag, "collapse inner term must be a block")
         if n > #acc then return inner_term end
         local it = acc[n]
         local inside = collapse (n+1, inner_term)
         assert (not inside.tag, "collapse must produce a block")
         if "Op" == it.tag then 
            -- [it] is a test, put it in an [if ... then .. end] statement
            return +{block: if -{it} then -{inside} end }
         else 
            -- [it] is a statement, just add it at the result's  beginning.
            assert ("Let" == it.tag or "Local" == it.tag)
            return { it, unpack (inside) }
         end
      end
\end{Verbatim}

To fully understand this function, one must remember that test
operations are translated into {\tt`Op\{ <opname>, <arg1>, <arg2>\}}
AST. That's why we didn't have to put an explicit reminder in the
accumulator, remembering whether a quote was a test or a statement:
we'll just check for \verb|`Op|'s presence.

\subsubsection{Match statement compilation}
We already know how to translate a single pattern into a Lua test
statement. This statement will execute a given block, with all pattern
variables bound appropriately, if and only if the tested term matches
the pattern.

To build a complete match statement, we need to test all patterns in
sequence. When one of them succeeds, we must skip all the following
cases. We could do that with some complex ``\verb|else|'' parts in the
tests, but there is a simpler way to jump out of some nested blocks:
the ``\verb|break|'' statement. Therefore, we will enclose the cases
into a loop that executes exactly once, so that a ``\verb|break|''
will jump just after the last case. Then, we will add a
``\verb|break|'' at the end of every statement that doesn't already
finish with a ``\verb|return|'', so that other cases are skipped upon
successful matching. To sum up, we will translate this:

\begin{Verbatim}[fontsize=\scriptsize]

match <foo> with
| <pattern_1> -> <x1>
| <pattern_2> -> <x2>
  ...
| <pattern_n> -> <xn>
end
\end{Verbatim}

\noindent into this:

\begin{Verbatim}[fontsize=\scriptsize]

repeat
  local v1, v2, ... vx -- variables used to store subpatterns
  let v1 = <tested_term>
  if <compilation of pattern_1> ... then x1; break end
  if <compilation of pattern_2> ... then x2; break end
  ...
  if <compilation of pattern_n> ... then xn; break end
until true
\end{Verbatim}

First, we add final \verb|break| statements where required, and we
compile all (pattern, block) pairs:

\begin{Verbatim}[fontsize=\scriptsize]

      -------------------------------------------------------------------
      -- parse all [pattern ==> block] pairs. Result goes in [body].
      -------------------------------------------------------------------
      local body = { }
      for _, case in ipairs (cases) do
         acc = { } -- reset the accumulator
         parse_pattern (1, case[1], var(1)) -- fill [acc] with conds and lets
         local last_stat = case[2][#case[2]]
         if last_stat and last_stat.tag ~= "Break" and 
            last_stat.tag ~= "Return" then
            table.insert (case[2], `Break) -- to skip other cases on success
         end
         local compiled_case = collapse (1, case[2])
         for _, x in ipairs (compiled_case) do table.insert (body, x) end
      end
\end{Verbatim}

\noindent Then, we can just splice it into the appropriate quote:

\begin{Verbatim}[fontsize=\scriptsize]

      -------------------------------------------------------------------
      local local_vars = { }
      for i = 1, max_n do table.insert (local_vars, var(i))  end
      
      -------------------------------------------------------------------
      -- cases are put inside a [repeat until true], so that the [break]
      -- inserted after the value will jump after the last case on success.
      -------------------------------------------------------------------
      local result = +{ stat: 
         repeat
            -{ `Local{ local_vars, { } } }
            (-{var(1)}) = -{tested_term}
            -{ body }
         until true }
      return result
      -------------------------------------------------------------------
\end{Verbatim}

There is one point to notice in this quote: \verb|body| is used where
a statement is expected, although it contains a {\em list} if
statements rather than a single statement. Metalua is designed to
accept this, i.e. if {\tt a, b, c, d} are statements, AST {\tt
`Do\{  a, b, c, d \}} and {\tt`Do\{ a, \{ b, c \}, d\} } are
equivalent. This feature partially replaces Lisp's \verb|@(...)|
operator.

\subsubsection{Syntax extension}
To use this, we provide a syntax inspired by OCaml\footnote{It is
  actually the same syntax as OCaml's, except that we introduced an
  explicit {\tt end} terminator, to stay homogeneous with Lua.}: 

\begin{Verbatim}[fontsize=\scriptsize]

match <foo> with
  <pattern> -> block
| <pattern> -> block
  ...
| <pattern> -> block
end
\end{Verbatim}

For this, we need to declare new keywords \verb|match|,
\verb|with| and \verb|->|. Then, we build the (pattern, block) parser
with \verb|gg.sequence{ }|, and read a list of them, separated with
``\verb+|+'', thanks to \verb|gg.list{ }|. Moreover, we accept an
optional ``\verb+|+'' before the first case, so that all cases line up
nicely:

\begin{Verbatim}[fontsize=\scriptsize]

----------------------------------------------------------------------
mlp.lexer:add{ "match", "with", "->" }

mlp.stat:add{ "match", mlp.expr, "with", gg.optkeyword "|",
              gg.list{ gg.sequence{ mlp.expr, "->", mlp.block },
                       separators  = "|",
                       terminators = "end" },
              "end",
              builder = |x| match_parser (x[1], x[3]) }
----------------------------------------------------------------------
\end{Verbatim}

\noindent Now, if you try this\ldots\ it won't work! Indeed, Metalua
needs to know what keywords might terminate a block of statements. In
this case, it doesn't know that ``\verb+|+'' can terminate a block. We
need therefore to add the following statement:

\begin{Verbatim}[fontsize=\scriptsize]

mlp.block.terminators:add "|"
\end{Verbatim}

\noindent Finally that's it, we have implemented a working pattern
matching system in 75 lines of code!

\subsubsection{Possible improvements}
Here are a couple of suggestions to further improve the pattern
matching system presented above. Some of these proposals can be
implemented very quickly, some others more complex; all of them
present some practical interest.

The code of the basic version, as presented here, is available at
\url{http://metalua.luaforge.net/FIXME}.

\paragraph{Booleans} Boolean constants aren't handled in the system
above, neither as table keys nor as patterns. They're easy to add, and
doing it will help you get gently into the code.

\paragraph{Gotos considered beneficial} Gotos might be harmful in
hand-written programs, but they're a bliss for machine-generated
code. They would slightly simplify the code of pattern matching as
presented above; but for many extension proposals listed below, they
will make reasonnably easy some things which would otherwise be
awfully contrived. Exercice: simplify the implementation above as much
as possible by using gotos.

Labels and gotos in metalua ASTs are represented as {\tt`Label\{ id
  \}} and {\tt`Goto\{ id \}} respectively, with {\tt id} an
identifier, typically generated by {\tt mlp.gensym()}. It is always
safe to jump out of a block; jumping into a block is not guaranteed
against weird interactions with local variables and upvalues.

\paragraph{{\tt collapse()} optimization} Instead of nesting if
statements systematically, two nested {\tt if}s without {\tt else}
branches can be simplified in a single branch with an {\tt and}
operator. Not sure it would change the bytecode's efficiency, but
that's a good exercice of AST manipulation.

\paragraph{Superfluous assignments} When parsing a table entry, we
assign it to a variable, then recursively call {\tt parse\_pattern()}
on it; in the frequent case where the entry was simply a variable, it
re-assigns it again. This could be optimized away.

\paragraph{Checking table arity} In many cases, it is practical to
check the number of elements in the array-part of the table. Here is a
simple modification proposal: by default, a table pattern matches a
table term only if they have the same number of array-part
elements. However, if the last element of the pattern is {\tt`Dots}
(a.k.a. {\tt+\{...\}}), then the term simply has to have 
{\it at least} as many array-part elements as the pattern.

\paragraph{Adding guards}
It is sometimes desirable to add arbitrary conditions for a pattern to
match, conditions which might no be expressed by a pattern. OCaml
allows to add them with a ``\verb|when|'' keyword:
\begin{Verbatim}[fontsize=\scriptsize]

match n with
| 0               -> print "zero"
| n when n%2 == 0 -> print "even number"
| _               -> print "odd number"
end
\end{Verbatim}
I'd advise you to prefer {\tt if} as a dedicated keyword, rather than
{\tt when}: it's unambiguous in this context, and saves a keyword
reservation. 

\paragraph{More bindings}
The way pattern matching is currently implemented, one can either bind
a subterm to a variable, or check its structure against a sub-pattern,
not both simultaneously. OCaml provides an ``\verb|as|'' operator,
which allows to do both (Haskell calls it ``\verb|@|''). For instance,
in the following example, any ADT whose tag is \verb|"RepeatMe"| will
be replaced by two occurrences of itself, while others will remain
unchanged:
\begin{Verbatim}[fontsize=\scriptsize]

match something with
| `RepeatMe{ ... } as r -> { r, r }
| x -> x
end
\end{Verbatim}
``\verb|as|'' will have to be declared as an infix operator, whose
meaning will remain undefined in expressions which are not patterns.

As an alternative, you can reuse an existing infix operator, thus
avoiding to mess the expression parser. For instance, use {\tt *}
instead of {\tt as}. You can go further, and implement {\tt +} as an
``or'' operator ({\tt pattern1 + pattern2} would match if either
of the patterns matches), although this would significantly complicate
the implementation of {\tt parse\_pattern()}. 

The {\tt+} operator might prove tricky to implement, if you don't
convert your code generator to gotos and labels first.

\paragraph{Linear bindings}
We should check, when compiling a pattern, that it is left-linear,
i.e. that variables don't appear more than once in the pattern. People
might be tempted to write things like this to check whether a tree is
symmetric:
\begin{Verbatim}[fontsize=\scriptsize]

match t with
| `Tree{ x, x } -> print "Symmetric!"
| `Tree{ x, y } -> print "Not symmetric"
| `Leaf{ _ }    -> print "Symmetric!"
end
\end{Verbatim}
However, this would work in Prolog but not with our pattern matching,
as two occurences of the same variable in the pattern don't cause an
equality test to be added. We should detect such non-linear variables,
and implement a suitable reaction:
\begin{itemize}
\item throw an error, or at least a warning;
\item or add an equality test between the terms matched by the
  non-linear variable;
\item or offer a syntax extension that lets the user provide his own
  equality predicate.
\end{itemize}

Notice that the latter choice would drive us towards a Prolog
unification algorithm, which opens interesting opportunities.

You might offer an exception for variable ``{\tt\_}'', which is often
intended as a dummy, unused variable. Non-linear occurences of it
should probably be silently accepted, without even performing the
corresponding binding. 

\paragraph{Generalized assignments}
Yet another OCaml-inspired feature: assignments such as
``\verb|foo = bar|'', is almost a special
case of pattern matching with only one case: the left-hand side is
the pattern, and the right-hand side is the ``raw'' ``\verb|foo=bar|''
assignment. Seen this way, it allows to write things such as
``\verb|`If{ cond, block } = some_ast }|'' to assign \verb|cond| and
\verb|block| to the subparts of \verb|some_ast| (if we know that
\verb|some_ast| is the AST of an \verb|if| statement). 

If you go this way, however, make sure that the code generated for
simple {\tt let}s is as efficient as before! Moreover, there is an (easy) 
scoping issue: the variables assigned belong to the scope of the
surrounding block.

\paragraph{Pattern matchings as expressions}
Pattern matching are currently statements, and take statements as
right-hand sides of cases. We could allow pattern matchings where
expressions are expected: these would take expressions instead of
statements as right-hand sides. Two ways to implement this: the dirty
one (hack with functions to change match statements into expressions),
and the clean one (refactoring existing code, so that it is agnostic
about its right-hand side type, and provide two specialized
versions of it for statements and expressions).

\paragraph{Bootstrap it}
That's something language designers love to do, for largely mystic
reasons: writing a language's compiler in the language itself. Here,
the idea is to re-implement the pattern matching extension by using
pattern matching, and compile it with the older version. Comparing the
firsrt and second versions of the code will give you an idea of how
much code clarification is brought to you by the pattern matching
extension.

\paragraph{Pattern conjunction} Another feature to take from OCaml is
multiple patterns for a single block. Instead of associating one
block with one pattern, cases associate a block with a (non-empty)
list of patterns. All of these patterns have to bond the same
variables, except for {\tt\_}. The first pattern in the list to match
the tested term does the binding. Patterns are separated by
``\verb+|+''. Example:
\begin{Verbatim}[fontsize=\scriptsize]

match x with
| 1 | 2 | 3 -> print(x)
| n -> print "more than 3"
end
\end{Verbatim}
(Hint: put the block in a local function. $2^{\mathrm{nd}}$ hint: sort
bound variables, e.g. by lexicographic order. Or much simpler and
more effective: convert your code generator to gotos+labels first).

\paragraph{XML munching} Ever tried to transform some XML document through
XSLT? Did you feel that this was even more kludgy than XML itself? Here
is a challenging proposal:
\begin{itemize}
\item Realize, if you didn't already, that Metalua's ADT are
  isomorphic to XML, if you identify string-keys in tables with
  attributes, and limit there content to strings and number. For
  instance, ``{\tt <foo bar=3><baz/>eek</foo>}'' easily maps to ``{\tt
    `foo\{ bar=3, `baz, "eek" \}}'';
\item compare what ML-style pattern matching does with what XSLT
  does (and with what you'd like it to do);
\item design, implement, publish. You might want to google
  ``CDuce''\footnote{\url{http://www.cduce.org}} for neat ideas.
\end{itemize}

If you do this, I'd be really interested to put back your contribution
in the next version of Metalua!

\subsubsection{Correction}
Most of the improvements proposed here are actually implemented in the
{\tt match} library provided with metalua. Check its (commented)
sources!

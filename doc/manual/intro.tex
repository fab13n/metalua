\section{Concepts}

\paragraph{Lua}
Lua\footnote{\url{http://www.lua.org}} is an very clean and powerful
language, with everything the discrimiating hacker will love: advanced
data structures, true function closures, coroutines (a.k.a collaborative
multithreading), powerful runtime introspection and metaprogramming
abilities, ultra-easy integration with C.

The general approach in Lua's design is to implement a small number of
very powerful concepts, and use them to easily offer particular
services. For instance, objects can be implemented through metatables
(which allow to customize the behavior of data structures), or through
function closures. It's quite easy to develop a class based system
with single or multiple inheritance, or a prototype based system 
{\`a la} Self\footnote{\url{http://research.sun.com/self}}, or
the kind of more advanced and baroque things that only CLOS users
could dream of...

Basically, Lua could be though of as Scheme, with:
\begin{itemize}
\item a conventional syntax (similar to Pascal's or Ruby's);
\item the associative table as basic datatype instead of the list;
\item no full continuations (although coroutines are actually one-shot
  semi-continuations);
\item no macro system.
\end{itemize}

\paragraph{Metalua}
Metalua is an extension of Lua, which essentially addresses the lack
of a macro system, by providing compile-time metaprogramming (CTMP)
and the ability for user to extend the syntax from within Lua.

Runtime metaprogramming (RTMP) allows a program to inspect itself
while running: an object can thus enumerate its fields and methods,
their properties, maybe dump its source code; it can be modified
on-the-fly, by adding a method, changing its class, etc. But this
doesn't allow to change the shape of the language itself: you cannot
use this to add exceptions to a language that lacks them, nor
call-by-need (a.k.a. ``lazy'') evaluation to a traditional language,
nor continuations, nor new control structures, new operators... To do
this, you need to modify the compiler itself. It can be done, if you
have the sources of the compiler, but that's generally not worth it,
given the complexity of a compiler program and the portability and
maintenance issues that ensue.

\paragraph{Metaprogramming}
A compiler is essentially a system which takes sources (generally as
a set of ASCII files), turn them into a practical-to-play-with data
structure, does stuff on it, then feeds it to a bytecode or machine
code producer. The source and byte-code stages are bad abstraction
levels to do anything practical: the sensible way to represent code,
when you want to manipulate it with programs, is tha abstract syntax
tree (AST). This is the practical-to-play-with abstraction level
mentionned above: a tree in which each node corresponds to a control
structure, where the inclusion relationship is respected (e.g. if an
instruction I is in a loop's body B, then the node representing I
is a subtree of the tree representing B)...

CTMP is possible if the compiler allows its user to read, generate and
modify AST, and to splice these generated AST back into programs. This
is done by Lisp and Scheme by making the programmer write programs
directly in AST (hence the lot of parentheses in Lisp sources), and by
offering a magic instruction that executes during compilation a piece
of code which generates an AST, and inserts this AST into the source
AST: that magic couple of instructions is the macro system.

Metalua has a similar execute-and-splice-the-result magic construct;
the main difference is that it doesn't force the programmer to
directly write in AST (although he's allowed to if he finds it most
suitable for a specific task). However, supporting ``real language
syntax'' adds a couple of issues to CTMP: there is a need for
transformation from real syntax to AST and the other way around, as
well as a need for a way to extend syntax.

This manual won't try to teach Lua, there's a wealth of excellent
tutorials on the web for this. I highly recommand Roberto
Ierusalimschy's ``Programming in Lua''
book\footnote{Programming in Lua, 2nd edition.\\
  Published by Lua.org, March 2006\\
  ISBN 85-903798-2-5 Paperback, 328 pages\\
  Distributed by Ingram and Baker \& Taylor.}, 
a.k.a. ``the blue PiL'', probably one of the best programming books
since K\&R's ``The C Language''. Suffice to say that a seasonned
programmer will be able to program in Lua in a couple of hours,
although some advanced features (coroutines, function environments,
function closures, metatables, runtime introspection) might take
longer to master if you don't already know a language supporting them.

Among resources available online, my personal favorites would be:
\begin{itemize}
\item The reference manual: \url{http://www.lua.org/manual/5.1}
\item The first edition of PiL, kindly put online by its author at
  \url{http://www.lua.org/pil}
\item A compact reference sheet (grammar and standard libraries) by
  Enrico Colombini: \url{http://lua-users.org/wiki/LuaShortReference}
\item Generally speaking, the Lua community wiki
  (\url{http://lua-users.org/wiki}) is invaluable.
\item The mailing list (\url{http://www.lua.org/lua-l.html})
  and the IRC channel (\url{irc://irc.freenode.net/#lua}) are
  populated with a very helpful community.
\item You will also find a lot of useful programs and libraries for
  Lua hosted at \url{http://luaforge.net}: various protocol parsers, 
  bindings to 2D/3D native/portable GUI, sound, database drivers...
\item A compilation of the community's wisdom will eventually be
  plubished as ``Lua Gems''; you can already check its ToC at
  \url{http://www.lua.org/gems}
\end{itemize}

So, instead of including yet another Lua tutorial, this manual will
rather focus on the features specific to Metalua, that is mainly:
\begin{itemize}
\item The couple of syntax extensions offered by Metalua over Lua;
\item The two CTMP magic constructs \verb|+{...}| and \verb|-{...}|;
\item The libraries which support CTMP (mainly for syntax extension).
\end{itemize}

\paragraph{Metalua design philosophy}
Metalua has been designed to occupy a vacant spot in the space of
CTMP-enabled languages:
\begin{itemize}

\item Lisp offers a lot of flexibility, at the price of macro-friendly
  syntax, rather than user-friendly. Besides the overrated problem of
  getting used to those lots of parentheses, it's all too tempting to
  mix macros and normal code in Lisp, in a way that doesn't visually
  stand out; this really doesn't encourage the writing of reusable,
  mutually compatible libraries. As a result of this extreme
  flexibility, large scale collaboration doesn't seem to happen, and
  Lisps lack a {\em de facto} comprehensive set of standard libs,
  besides those included in Common Lisp's specification. Comparisons
  have been drawn between getting Lispers to work together and herding
  cats\ldots

\item Macro-systems bolted on existing languages (Template
  Haskell\footnote{\url{http://www.haskell.org/th/}},
  CamlP5\footnote{\url{http://pauillac.inria.fr/~ddr/camlp5/}},
  MetaML\footnote{\url{http://www.cse.ogi.edu/pacsoft/projects/metaml}}\ldots)
  tend to be hard to use: the syntax and semantics of these target languages are
  complex, and make macro writing much harder than necessary. Moreover, for some
  reason, most of these projects target statically typed languages: although
  static inference type systems {\`a la} Hindley-Milner are extremely powerful
  tools in many contexts, my intuition is that static types are more of a burden
  than a help for many macro-friendly problems.

\item Languages built from scratch, such as
  converge\footnote{\url{http://convergepl.org}} or
  Logix\footnote{\url{http://http://www.livelogix.net/logix/index.html}}, have
  to bear with the very long (often decade) maturing time required by a
  programming language. Moreover, they lack the existing libraries and
  developpers that come with an already succesful language.

\end{itemize}

\noindent Lua presents many features that beg for a real macro system:

\begin{itemize}
\item Its compact, clear, orthogonal, powerful semantics, and its
  approach of giving powerful generic tools rather than ready-made
  closed features to its users.
\item Its excellent supports for runtime metaprogramming.
\item Its syntax, despite (or due to its) being very readable and easy
  to learn, is also extremely simple to parse. This means no extra
  technology gets in the way of handling syntax (no BNF-like
  specialized language, no byzantine rules and exceptions). Even more
  importantly, provided that developers respect a couple of
  common-sense rules, cohabitation of multiple syntax extensions in a
  single project is made surprizingly easy.
\end{itemize}

Upon this powerful and sane base, Metalua adds CTMP with the following
design goals:

\begin{itemize}
\item Simple things should be easy and look clean: writing simple
  macros shouldn't require an advanced knowledge of the language's
  internals. And since we spend 95\% of our time {\em not} writing
  macros, the syntax should be optimized for regular code rather than
  for code generation.
\item Good coding practices should be encouraged. Among others,
  separation between meta-levels must be obvious, so that it stands
  out when something {\em interesting} is going on. Ideally, good code
  must look clean, and messy code should look ugly.
\item However, the language must be an enabler, not handcuffs: it
  should ensure that users know what they're doing, but it must
  provide them with all the power they're willing to handle.
\end{itemize}

Finally, it's difficult to talk about a macro-enabled language without making
Lisp comparisons. Metalua borrows a lot to Scheme's love for empowering
minimalism, through Lua. However, in many other respects, it's closer to Common
Lisp: where Scheme insists on doing The Right Thing, CL and metalua assume that
the programmer knows better than the compiler. Therefore, when a powerful but
potentially dangerous feature is considered, metalua generally tries to warn the
user that he's entering the twilight zone, but will let him proceed. The most
prominent example is probably macro hygiene. Scheme pretty much constraints
macro writing into a term rewriting system: it allows the compiler to enforce
macro hygiene automatically, but is sometimes crippling when writing complex
macros (although it is, of cource, Turing-complete). Metalua opts to offer CL
style, non-hygienic macros, so that AST are regular data manipulated by regular
code. Hygienic safety is provided by an {\em optional} library, which makes it
easy but not mandatory to do hygienic macros.

\section{{\tt clist}: Lists by comprehension} 

This extension offers improved tables-as-list syntax:
\begin{itemize}
\item lists by comprehension;
\item literal lists splicing;
\item list sub-sampling.
\end{itemize}

Lists by comprehensions allow to describe a list in terms of
generation loops and filtering. The loops are the two flavors of
``for'' controls, and the syntax is {\tt\{ <value> <loop\_header>
  \}}. For instance, the list {\tt\{10, 20, 30, 40, 50\}} can be generated
as {\tt\{ x for x=10, 50, 10 \}}, or {\tt\{ 10*x for i=1, 5 \}}. The
list of all keys from table {\tt t} can be gathered with {\tt\{ k for
  k, v in pairs(t) \}}. Several loops can be nested. For instance, the
list of all products of elements taken from list {\tt a} and {\tt b}
can be computed with {\tt\{ i*j for \_, i in ipairs(a) for \_, j in
  ipairs(b) \}}.

Finally, results can be filtered out of a list. For instance, the
elements of {\tt a} which are multiple of 3 can be gathered with
{\tt\{ x for \_, x in ipairs(a) if x\%3==0 \}}.

In Lua, when a list is defined and one of its elements is a multiple
return function, only the first returned value is kept, except for the
last element (cf. Lua manual 2.5.7). For instance, in the example
below, {\tt x} is set to {\tt\{1, 1, 1, 2, 3\}}: only the last call to
{\tt f()} is expanded:

\begin{verbatim}
function f()
  return 1, 2, 3
end
x = { f(), f(), f() }
\end{verbatim}

The extension offers a way to expand intermediate response: they ahve
to be followed by {\tt...}. In the example above, {\tt y = \{f()...,
  f()..., f()...\}} would expand as {\tt\{1, 2, 3, 1, 2, 3, 1, 2,
  3\}}.

Comprehensions are naturally expanded, i.e. another way to write {\tt
  y} would have been {\tt y = \{i for i=1,3; i for i=1,3; i for
  i=1,3\}} (notice however that we had to separate elements with
semicolons rather than commas: if we didn't, the {\tt i} of the second
loop would have been tekane as a third parameter to the first for loop
header).

Sub-sampling is done with indexes, by using the comma to separate
indices, and {\tt...} as an infix operator to denote intervals. The
latter binds tighter than the former. For instance:

\begin{verbatim}
x = { i for i=101, 130 }
y = x[1 ... 10, 20, 25]
z = { i for i=101,110; 120; 125 }

assert (#y == #z)
for i = 1, #x do
  assert (y[i] == z[i]
end
\end{verbatim}

Beware of a lexing issue: if you write ``{\tt[1...n]}'', it will be
interpreted as number {\tt 1.} followed by operator {\tt..}: put a
space between literal numbers and operators starting with a dot.

Notice taht there are now two substancially different operators with
very similar syntaxes: the original index operator, which returns a
single element, and the sub-sampling operators, which returns a list
of elements. If you want to returna single element list, you can
either reconstruct it from the regular index operator ({\tt
  y=\{x[i]\}}), or use a single element wide interval ({\tt
  y=x[i...i]}).

FIXME: there should be \verb|x[...i]| and \verb|x[i...]| sub-sampling
notations, but they aren't currently implemented.
%-*-mode:latex; eval:(whizzytex-mode);-*-
%; whizzy-master tilo.tex

\section{Static type system}

In the previous section, we've seen that we wanted a type system to
prevent indexing of non-tables as well as application of
non-functions. We've also mentioned that a type system which
eliminates all incorrect terms will also eliminate some correct ones
(a direct consequence of the calculus' T\"uring-completeness, easily
demonstrated by encoding the $\lambda$-calculus in it). This section
will try to find a reasonable compromise: a type system which accepts
a lot of ``reasonable'' terms, catches all incorrect ones, and doesn't
force too much book-keeping on users.

\subsection{Notations}
%\begin{itemize} \item 
$<:$ denotes subtyping. $\T T_1 <: \T T_2$ means that
$\T T_1$ is a subtype of $\T T_2$, i.e. a term of type $\T T_1$ can
be used everywhere a term of type $\T T_2$ is expected.
%\end{itemize}

\subsection{Type System}

$$\begin{array}{rcll}
\T E &::=& [\overline{P:\T F}| \T F]  &\eqlabel{TE-Table}\\
&|& \T P & \eqlabel{TE-Primitive}\\
&|& \bar\T E \rightarrow \bar\T E &\eqlabel{TE-Function}\\
%&|& v &\eqlabel{TE-Variable} \\
&|& \top &\eqlabel{TE-Top} \\
\\
\T P &::=& \Tnil\ |\ \Tboolean\ |\ \Tnumber\ |\ \Tstring\\
\\
\T F &::=&  \J{\T E} &\eqlabel{TF-Just} \\
&|&  \C{\T E} &\eqlabel{TF-Currently} \\
&|& \V{\T E} &\eqlabel{TF-Var} \\
&|& \K{\T E} &\eqlabel{TF-Const} \\
&|& \F &\eqlabel{TF-Field} \\
\\
\T S &::=& \Return{\bar\T E} &\eqlabel{TS-Return}\\
&|& \Tnoreturn &\eqlabel{TS-None}\\
\end{array}$$


\paragraph{Lua key features to respect}

The type system should capture as many Lua-specific idioms as
possible. Tables are of course central in Lua, and quite similar in
some respects to objects of calculi such as Abadi \&
Cardelli's~\cite{sigma}, Fisher Honsell \& Mitchell~\cite{fhm}, or
R\'emy's~\cite{remy}. Among others, they are defined with no native
notion of classes. The most striking Lua-specific features are:

\begin{itemize}
\item Lua tables---and therefore Lua itself---are deeply
  imperative. There are primitives in Lua to alter a table, but
  neither to copy nor to functionally update it; despite closures and
  tail-call optimization, idiomatic Lua code is not functional. This
  contrasts with most theoretical studies of object-oriented calculi,
  which inherit from $\lambda$-calculus a preference for functional
  primitives and idioms.
\item Lua tables are arbitrary value$\rightarrow$value hashtables,
  whereas object calculi typically index object fields with labels
  taken from a separate (enumerable) set. In the first version of our
  type system, we'll only type tables whose keys have primitive types
  \verb+string+, \verb+number+ or \verb+boolean+. Homogeneously typed
  hashtables should be easy to add at a later stage---they mostly
  behave as simplified functions, type-wise; but arbitrarily mixed
  tables, acting as a raw mix of records and hashtables, are untypable
  in general.
\item Lua makes a distinction between statements and expressions. This
  means that two distinct type kinds $\T S$ and $\T E$ are
  defined. This is again in contrast with most theoretical calculi,
  which are purely expression-based.
\item There's no notion of ``undefined label'': all table keys except
  \verb+nil+ are defined in all tables; keys which haven't been
  explicitly assigned are associated with the value \verb+nil+.
\item A key consequence is that an table's type changes during its
  lifetime: when created all its fields have type \verb+nil+, then
  these field types are modified as meaningful values are put in
  them. Whereas most calculi allow to add fields to object, ours will
  allow to change their type.
\item Lua functions take multiple arguments, and return multiple
  values.
\item Expressions can be collected into expression sequences, to be
  used in assignments, function calls and function results. They get
  their dedicated type kind $\bar\T E$.
\end{itemize}

Beyond those specific needs, the type system will include many staples
of modern type systems, such as structural subtyping, functions
contravariant in their arguments and covariant in their results,
covariant read-only table fields, invariant read+write fields.

Types will not be nullable (there won't be any legal way to derive
$\Nil<:\T E$ for any $\T E$ other than \Nil): catching ``NPE'' ({\em
  Null Pointer Exceptions}, as nicknamed by traumatized Java
developers) has been done for ages in ML-inspired languages; Hoare
himself, who first introduced implicitly nullable types in Algol,
called them ``{\em my billion dollar mistake}''~\cite{?}.  However,
since nullable function arguments and results are an important idiom
in Lua, future versions of the calculus will have to support either an
explicit \verb+nullable+ type modifier, or a generic union type, allowing
to type e.g. an optional number as \verb+nil|number+.

\paragraph{Tables}
Lua tables accept all values as keys except \Nil, and there's no
notion of unset/undefined key; it's legal to request {\tt foo["bar"]}
even if the key {\tt"bar"} has never been set in {\tt foo}: it will
return \Nil. So in practice, all but a finite set of keys in a given
table will return a value other than \Nil. To reflect this in the type
system, a table type will contain a default field type, shared by all
but a finite number of its fields; this type will usually be set to
some variant of \Nil. The other key/type pairs are listed explicitly.
For instance, $[\texttt{"x"}:\K{\Tnumber}|\K\Nil]$ describes a table
such as \verb+{["x"]=1}+, with a field {\tt"x"} of type $\K\Tnumber$,
and all other fields left to \Nil.

In the current version, field keys are limited to values of atomic
types \verb+string+, \verb+number+ or \verb+boolean+. Tables using
other keys will not be typable statically. Some future extensions are
possible, and will be studied separately.

In contrast with type systems inspired by Abadi \& Cardelli's, we
won't introduce a $\zeta(s)$ self-type binder in the type
system~\cite{sigma}: it substantially complicates it, mostly to allow a
functional-style use of objects that doesn't seem to correspond to any
widespread Lua idiom.

Table types are considered equal modulo fields reordering, and
expansion of the default field type; the following types are all
considered equal:

$$\begin{array}{ll}
 \phantom{{}={}}
    [P_1: \T F_1; P_2:\T F_2|\T F_d]\\
 {}=[P_2: \T F_2; P_1:\T F_1| \T F_d] & \eqlabel{reordering}\\
 {}=[P_1: \T F_1; P_2:\T F_2; P_3:\T F_d| \T F_d]
    & \eqlabel{default expansion} \\
\end{array}$$

Moreover, it's illegal for a field key to appear more than once in the
same table: $[P_1:\T F_1; P_1:\T F_2 | \T F_d]$ is not a well-formed
type.

Finally, we'll admit as a shortcut that $[\overline{P:\T F}]$ means
$[\overline{P:\T F}|\F]$.

\paragraph{Field types}

Field types are expression types with a prefix modifier: $\J\T E$,
$\C\T E$, $\V\T E$, $\K\T E$, and simply $\F$ without a type
parameter. They give control over field variance, i.e. they prevent
some operations, but in exchange allow more permissive subtyping, and
hence allow to use tables in more contexts. \verb+var+, \verb+const+
and \verb+field+ will be familiar to people who studied structural
subtyping, and offer the expected variance properties. $\J\T E$ and
$\C\T E$ are more unusual, and allow to change a value's type for an
unrelated one, under specific conditions.


\subparagraph{Read-write fields}
$\V\T E$ is the type of a field which can be read and written with
values of type $\T E$. It's not covariant, i.e. even if $\T E_1<:\T
E_2$, we don't have $[P:\V\T E_1]<: [P:\V\T E_2]$. To see why, let's
consider the subtyping relationship $\texttt {positive} <: \texttt
{number}$ \footnote{Positive numbers are numbers, but not the other
  way around. This example of subtyping has been chosen because it's
  hopefully familiar and intuitive for everyone; but it's only
  intended to illustrate variance issues, and the calculus won't have
  a specific {\tt positive} type.}. If we had $[P:\V\texttt
    {positive}] <: [P:\V\texttt {number}]$, we could take a
table of type $[P:\V\texttt {positive}]$, partially
forget its type through subtyping into $[P:\V\texttt
    {number}]$, then write a negative number in its field
$P$. Other parts of the program, which retained the more precise
type $[ P: \V\texttt{positive}]$ for the same
table, might break because they take for granted that $P$'s
content is positive.

\subparagraph{Read-only fields}
If we promise not to overwrite a field, however, we can make it
covariant. This is the purpose of $\K\T E$ fields: the type system
will prevent from updating such fields, but in exchange, whenever we
have $\T E_1<:\T E_2$, we also get $\K\T E_1<: \K\T E_2$. Adapting the
previous example, $[P: \K\texttt {positive}]$ is a subtype of $[P:
  \K\texttt {number}]$, because when reading its $P$ field, one
gets a {\tt positive} which is indeed a {\tt number}; since we can't
write in it, there's no danger of putting a negative number where the
type system expects to find only positive ones.

\subparagraph{Contravariant fields}
Symmetrically, we could promise not to read a field, and get
contravariance in exchange (whenever $\T E_1<:\T E_2$, we get
$[P:\texttt{writeonly }\T E_2]<:[P:\texttt{writeonly }\T E_1]$). However,
this seems of limited practical use, so we'll leave this out of the
type system.

\subparagraph{All fields} If we promise neither to read nor
to write a given field, it becomes bivariant: whether $\T E_1<:\T E_2$
or $\T E_2<:\T E_1$, or even if $\T E_1$ and $\T E_2$ are
incomparable, we'd still have $\F\ \T E_1 <: \F\ \T
E_2$. We'll therefore simply write it $\F$: no need to keep $\T
E$ in it, since it isn't used anyway. It's the super-type of all other
type fields, and it acts as the {\tt private} modifier does in C++
inspired languages: you can neither override nor use a field of this
type.

\subparagraph{Type-changing field types}

$\C\T E$ means that a field currently has type $\T E$, but that this
type can be changed without breaking the program. This is an
unusually liberal typing rule, and as such, it will only be allowed
under strictly controlled circumstances. Most notably, a table type
with some \verb+currently+ fields will have to be used linearly: if
several variables allowed access to the same \verb+currently+ field, one
variable could change the field's content type without the other
variable's knowledge, and break the program in unpredictable
ways. Hence, it will be mandatory to weaken the type of $\C\T
E$ fields into \F, before using them in non-linear ways, thus
preventing both read and write operations on it.

$\C\T E$ field types are intended to allow idioms such as ``{\tt
  x=\verb+{ }+;x.f\_1= E1; ...; x.f\_n=En}''; the type of \verb+x+ in
this program will change at each statement of this sequence, from
$[|\C\Nil]$ to $[\texttt{"f\_1"}:\C\T E_1; ...; \allowbreak
  \texttt{"f\_n"}:\C\T E_n|\C\Nil]$.

We mentioned a criterion of linearity: there must be at most one
reference to a $\C\T E$ field. Otherwise, one reference might change
the type of the field's content without the other references'
knowledge. The typing rules will be designed in such a way that
whenever extra references to it are created, these will be typed as
\F, i.e. inaccessible.

\subparagraph{Unreferenced field types}

Finally, we need a type indicating that an object is completely
unreferrenced, and can therefore be stored safely into a $\C\T E$
field. For instance, in {\tt x=\{foo=1\}}, if the right-hand-side
table was typed $[\texttt{"foo"}:\C\Tnumber| \allowbreak \C\Nil]$, it
would have to be weakened into $[|\F]$ before being stored in $x$, in
case there was already a reference to it. 

Therefore, we distinguisg $\C\T E$ the type of a field referenced
once, and $\J\T E$ the type of a field which isn't referenced at
all. In the example above, the right-hand-side is typed
$[\texttt{"foo"}:\J\Tnumber|\J\Nil]$, and weakened into
$[\texttt{"foo"}:\C\Tnumber| \allowbreak \texttt{cur}\-\texttt{rently
  }\Nil]$ when stored into \verb+x+. A further \verb+y=x+ statement
would see the type stored in \verb+y+ weakened to $[|\F]$.

\subsection{Subtyping rules}
The subtyping relationship is a partial order, defined as the smallest
transitive closure of the rules listed in this subsection.

\paragraph{Structural rules}
The subtyping relationship, defined over expression, field and
statement types, is reflexive and transitive; $\top$ is the biggest
expression type:

$$
\frac{}{
\T E <: \top
}\eqlabel{$<:\top$}
$$
$$
\frac{}{\T E<:\T E}
\qquad
\frac{}{\T F<:\T F}
\qquad
\frac{}{\T S<:\T S}
\qquad
\eqlabel{$<:$Refl}
$$
$$
\frac{\T E_1<:\T E_2 \quad \T E_2 <: \T E_3}{\T E_1<:\T E_3}
\quad
\frac{\T F_1<:\T F_2 \quad \T F_2 <: \T F_3}{\T F_1<:\T F_3}
\quad
\frac{\T S_1<:\T S_2 \quad \T S_2 <: \T S_3}{\T S_1<:\T S_3}
$$
$$
\eqlabel{$<:Trans$}
$$


\paragraph{Fields subtyping}

We have $\V\T E<:\K\T E$, \verb+const+'s covariance, and
\F\ the top field type:

$$
\frac{}{
{\T F} <: \F
}\eqlabel{$<:$Field}
$$
$$
\frac{}{
\V{\T E} <: \K{\T E}
}\eqlabel{$<:$Const}
%
\qquad
%
\frac{
\T E_1 <: \T E_2
}{
\K{\T E_1} <: \K{\T E_2}
}\eqlabel{$<:$Const${}^+$}
$$
$$
\frac{}{
\J\T E<:\C\T E}
\eqlabel{$<:$Currently}
\qquad
\frac{}{
\J\T E<:\V\T E}
\eqlabel{$<:$Var}
$$
$$
\frac{
\T E_1 <: \T E_2
}{
\J{\T E_1} <: \J{\T E_2}
}\eqlabel{$<:$Just${}^+$}
$$

We do {\em not} have $\C\T E<:\V\T E$. Indeed, mutable field types are
not a special case of variable fields: the latter can be used with
less restrictions when linearity cannot be guaranteed. For instance,
if $\texttt{x}: [\V\Tnumber]$, its content can be assigned to \verb+y+
with ``\verb+x=y+'', and \verb+y+ will also have type $[\V\Tnumber]$.
However, if \verb+x+ had type $[\C\Tnumber]$, \verb+y+ would only get
type $[\F]$, because the following statement might be e.g. ``{\tt
  x=false}'': one cannot count on \verb+y+ keeping its \Tnumber\ type.

\paragraph{Functions subtyping}
Functions are contravariant in their arguments, and covariant in their
results:
$$
\frac{
\bar\T E_?^2 <: \bar\T E_?^1
\quad
\bar\T E_!^1 <: \bar\T E_!^2
}{
\bar\T E_?^1 \rightarrow \bar\T E_!^1
<:
\bar\T E_?^2 \rightarrow \bar\T E_!^2
}\eqlabel{$<:$Function}
$$


\paragraph{Tables subtyping}

Subtyping between tables is directly lifted from field subtyping;
unreferenced tables, marked with a prime, can be considered as
regular table whenever suitable.

$$\frac{
  (\forall n\in[0...m])\
  \T F_n^a <: \T F_n^b
}{
  [(P_n:\T F_n^a)^{\forall n\in[1...m]}|\T F_0^a] <:
  [(P_n:\T F_n^b)^{\forall n\in[1...m]}|\T F_0^a]
}\eqlabel{$<:$Table}
$$

% probably no need for structurel subtyping inside linear tables.

\paragraph{Expression sequences subtyping}
Subtyping between expression sequences is only defined between
sequences of the same length. Typing rules will pad sequence with
\verb+nil+s on the right whenever appropriate:

$$\frac{ (\forall n\in[1...m])\ \T E_1^n <: \T E_2^n }{
  (\T E_1^n)^{\forall n\in[1...m]} <: (\T E_2^n)^{\forall
    n\in[1...m]} }\eqlabel{$<:\bar\T E$}$$

\paragraph{Statements subtyping}
Statement types are either $\Tnoreturn$, or of the form $\Return\bar\T
E$. In the latter case, subtyping is lifted from
expression sequences subtyping:

$$
\frac{
\bar\T E_1 <: \bar\T E_2
}{
\Return{\bar\T E_1} <:  \Return{\bar\T E_2}
}\eqlabel{$<:$Return}
$$


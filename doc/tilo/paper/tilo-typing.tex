%-*-mode:latex; eval:(whizzytex-mode);-*-
%; whizzy-master tilo.tex

\subsection{Typing rules}

This section gives the typing rules, which allow to determine the
belonging of terms to certain types. To do that, we'll enrich the
calculus with a couple of type annotations. We won't discuss the
possibility to algorithmically infer some of those.

\paragraph{Typing environments}

The rules exposed below use typing environments $\Gamma: v\mapsto\T
F$, functions from variables to field types (rather than, as one could
have expected, expression types). Indeed, being able to separate
constants from variables in the type system is valuable, and more
importantly, the linearity issues handled by the \verb+currently+
modifier occur with local variables as well as with table fields.

This section won't address variable shadowing
issues\footnote{i.e. homonymies, as in ``{\tt local x; x=function(x)
    return x end}'', where there are two distinct variables which both
  share the name {\tt x}.}: since we work on static terms which we
don't evaluate, an appropriate $\alpha$-renaming before typing can
ensure that no such shadowing occurs.

\paragraph{Notations}
\begin{itemize}
\item $\Gamma[v \leftarrow \T F]$ is the function which, to $v$,
  associates $\T F$, and to all other values $w\in\textrm{dom} (\Gamma)
  \backslash \{v\}$ associates $\Gamma(w)$.
\item This definition is extended homomorphically to sequences of
  variables and values $\Gamma [\bar v \leftarrow \bar\T F]$.
\item The empty environment, i.e. the function with an empty domain,
  is written $\varnothing_\Gamma$.
%% \item $\Gamma_1\cup\Gamma_2$ is the function defined over
%%   $\textrm{dom}(\Gamma_1) \cup \textrm{dom}(\Gamma_2)$ which, to $v$,
%%   associates $\Gamma_1(v)$ or $\Gamma_2(v)$, when
%%   $\textrm{dom}(\Gamma_1) \cap \textrm{dom}(\Gamma_2) =
%%   \varnothing_\Gamma$.
%%
%% \item $\T F_1\lhd\T E=\T F_2$ means that it's legal to store an
%%   expression of type $\T E$ in a field of type $\T F_1$, and that
%%   afterward, the field's type becomes $\T F_2$. If the final type
%%   doesn't matter, the notation can be abridged into $\T F_1\lhd\T E$.
%% \item $(\T E_1, P)\lhd\T E = \T E_2$ is a variant of the above,
%%   operating on expression types. It means that an object of type $\T
%%   E_1$ can support an update of its field $P$ with an expression of
%%   type $\T E$, and that the receiving object's type becomes $\T E_2$.
\item $\Gamma\vdash T:\T T$ means that under the assumptions in
  environment $\Gamma$, term $T$ has type $\T T$.
\item The notation $\Gamma\vdash E\therefore\T F$ describes the types
  of left-hand sides operands in ``$=$'' assignment statements; these
  variables and table fields must retain field types rather than
  expression types, because their field qualifier indicates whether
  and how they can be updated.  It means ``under the assumptions
  $\Gamma$, and in an assignment's left-hand side context, expression
  $E$ has type $\T F$''.
\item We'll refer to expressions which can syntactically appear on the
  left-hand-side of an assignments, and in a ``$\therefore$''
  judgment, as ``slots''. Those are local variables
  \eqlabel{E-Variable} and indexed tables \eqlabel{E-Index}.
%% \item $\T Q$ denotes operators which prefix expression types $\T E$ 
%%   with annotations \verb+currently+, \verb+var+, \verb+const+, or
%%   simply return field type \verb+field+.
\end{itemize}

\paragraph{Calculus extensions}
The untyped calculus is extended with a couple of annotations which
will allow to insert typing hints at appropriate places in programs.
Syntactically, typing annotations will make heavy use of the pound
$\#$ ascii character. This character, unused in Lua where type
annotations may occur, will hopefully make typing annotations stand
out visually, and make it easy to preprocess them out of a program's
sources, so that it can be used by other interpreters.

\begin{itemize}
\item statement $\#\Return{\bar{\T E}}; \bar S$ weakens the type of
  statements sequence $\bar S$ to statement type $\Return\bar\T E$.
\item $\Function{\overline{v\ \# \T E}}{\bar S}$ allows to give type
  annotations to function parameters, which are notoriously hard to
  infer effectively.
%% \item $\#v:\T F;\bar S$ allows to weaken the type of variable $v$ to
%%   $\bar F$ when typing the remaining of the statements sequence $\bar
%%   S$. This is especially useful to weaken \verb+currently+ fields into
%%   \verb+const+ or \verb+var+ ones, so that they can be used in
%%   non-linear conditions.
\item Left-hand sides of assignments take a type field, allowing to
  handle \verb+currently+ field type updates, and linearity
  issues: $\overline{L\ \#\T F} = \bar\T E$.
\end{itemize}
%
$$
\frac{}{\Sigma,(\#\Return{\bar\T E};\bar S) \Rightarrow \Sigma, \varnothing_S}
%
\quad
%
\frac{\Sigma_1, \Function{\bar v}{\bar S} \Rightarrow \Sigma_2, f}
{\Sigma_1, \Function{\overline{v\ \#\T E}}{\bar S} \Rightarrow \Sigma_2, f}
$$
$$
\frac{
\Sigma_1, (\Local \bar L=\bar E;\bar S) \evalsto \Sigma_2, S
}{
\Sigma_1, (\Local \overline{L\ \#\T F}=\bar E;\bar S) \evalsto \Sigma_2, S
}
$$

%%%% Wrong: it only allows to change #currently types.
%% Notice that a dummy assignment $L_1\ \#\T F_1=L_1$, while not
%% having any side effect, allows to retype the variable/field $L_1$,
%% provided that the environment $\Gamma$ allows it. In an actual
%% implementation of tha calculus, some syntax sugar for this use case
%% would most likely be desirable.

\paragraph{Expression sequences}
Expressions in Lua can evaluate into multiple values. When
concatenating expressions in a sequence, Lua only keeps the first
value of each expression's evaluation, except for the last one which
is expanded \eqlabel{EE-Sequence}.  For instance, if we have
$\texttt{a}:(\T E_a^1;\T E_a^2)$, $\texttt{b}:(\T E_b^1;\T E_b^2)$
and $\texttt{c}:(\T E_c^1;\T E_c^2)$, then the type of the sequence
$(\texttt{a}; \texttt{b}; \texttt{c})$ is $(\T E_a^1; \T E_b^1;
\T E_c^1; \T E_c^2)$. A noteworthy property is that the number of
elements in the type might be bigger than the number of expressions in
the sequence. Another is that appending \Nil types at the end of a
sequence type doesn't change it: {\tt number, number, nil} must be
treated as equal to {\tt number, number}. This will be ensured by
every rule effectively combining expression type sequences.

$$
\frac{
(\forall n\in[1...m]) \quad \Gamma\vdash E_n:(\T E_n^1;\bar\T E_n)
}{
\Gamma\vdash \bar E:(\T E_n^1)^{\forall n\in[1...m]};\bar\T E_m
}\eqlabel{TR-$\bar E$}
$$

\paragraph{Primitive expressions}

$$
\frac{}{\Gamma\vdash \langle\textrm{number}\rangle:\Tnumber}
\quad
\frac{}{\Gamma\vdash \langle\textrm{string}\rangle:\Tstring}
$$
$$
\frac{}{\Gamma\vdash \texttt{true}: \texttt{boolean}}
\quad
\frac{}{\Gamma\vdash \texttt{false}: \texttt{boolean}}
$$
$$\eqlabel{TR-P}$$

%% \todo{We might have to type t and f table/function references too, in
%%   order to get some subject reduction lemma. Not sure how to do that
%%   in big-step semantics, though, as I won't get any Felleisen-style
%%   syntactic criterion.}

\paragraph{Statement sequences} 
Statement sequences appear in function bodies. What we need to know
about them, besides the fact that they don't fail during evaluation,
is the type of the expression sequences they return. Therefore we have
two families of statement types: $\Tnoreturn$ for terms which don't
return, and $\Return{\bar\T E}$ for terms returning a sequence of
expressions of type $\bar\T E$.

$$
\frac{
\Gamma\vdash S: \Tnoreturn
\quad
\Gamma\vdash \bar S: \T S
}{
\Gamma\vdash (S;\bar S): \T S
}\eqlabel{TR-$\Tnoreturn$}
%
\qquad
%
\frac{
\Gamma\vdash S: \Return{\bar\T E}
}{
\Gamma\vdash (S;\bar S): \Return{\bar\T E}
}\eqlabel{TR-{\tt return}}
$$
%
$$
\frac{
\Gamma\vdash\bar S:\Return{\bar\T E_1}
\quad
\bar\T E_1 <: \bar\T E_2
}{
\Gamma\vdash(\#\Return{\bar\T E_2}; \bar S):\Return{\bar\T E_2}
}
\eqlabel{TR-\#{\tt return}}
$$
\paragraph{Variables}
Variable types are remembered in the environment $\Gamma$. They're
slots, and have a field type $\T F$ rather than an expression type $\T
E$. This allows to use them on the left-hand-side of assignements, to
remember whether they're constant, private, or whether their current
type can be updated. Field type judgments use the operator
``$\therefore$'', to avoid being confused with expression type
judgments using ``:''.

$$
\frac{
\Gamma(v) = \T F
}{
\Gamma \vdash v \therefore \T F
}
\eqlabel{TR-$\therefore$}
$$

When a slot $L$ is used as a normal expression rather than an
assignment's left-hand-side, its field type can be projected into an
expression type:

$$
\frac{
\Gamma \vdash L \therefore \C\T E
}{
\Gamma \vdash L:\T E
}
\quad
\frac{
\Gamma \vdash L \therefore \V\T E
}{
\Gamma \vdash L:\T E
}
\quad
\frac{
\Gamma \vdash L \therefore \K\T E
}{
\Gamma \vdash L:\T E
}
\quad\eqlabel{TR-L}
$$

There's no rule to project type \F: it's never legal to use such a
field in an expression's context. We'll see that no rule to project
\verb+just+ types is needed either, because there's no legal way to
derive a \verb+just+ type for a left-hand=side expression.

\paragraph{Table fields} 
In this rule, the variable $\phi$ denotes a set of key/field types,
plus the table's default type.

$$\frac{
\Gamma\vdash E_T: [P: {\T F_K};\phi]
}{
\Gamma\vdash E_T[P] \therefore \T F_K
}\eqlabel{TR-$[\ ]$}$$

In some cases, an expansion of the default field type might be needed,
e.g. we have ${\Gamma\vdash E:[|\K\Nil]} \over {\Gamma\vdash
  E[\texttt{"x"}] \therefore \K\Nil}$, because $[|\K\Nil] =
[\texttt{"x"}:\K\Nil|\K\Nil]$.

\paragraph{Literal tables}
Literal tables have all their fields typed with \verb+just+ modifiers.
When the literal table will be stored in a variables, the field types
will be weakened into either {\tt currently}, {\tt var}, {\tt const}
or {\tt field} types.

$$
\frac{
(\forall n\in[1...m]) \quad \Gamma\vdash E_n:(\T E_n^1;\bar\T E_n)
}{
  \Gamma\vdash \{([P_n] = E_n)^{\forall n\in[1...m]}\}:
  [(P_n:\J{\T E_n^1})^{\forall n\in[1...m]}|\J\Nil]
}$$
\vspace{-2em}
\begin{flushright}
  $\eqlabel{TR-Table}$
\end{flushright}
 


\paragraph{Local variable declarations}

Unlike in most type systems, newly created variables are given the
\Nil\ type, rather than the type of their future content. This is
because assignment statements change the type of the variables on
which they operate, as we'll see below. Besides, typing as non-\Nil\ a
variable while it contains \Nil\ wouldn't be sound.

$$
\frac{
\Gamma[\bar v\leftarrow\overline{\C\Nil}]\vdash S:\T S
}{
\Gamma\vdash (\Local{\bar v}; \bar S) : \T S
}\eqlabel{TR-Local}
$$

\paragraph{Assignments}

Assignments can change the type of \verb+currently+ variables and
fields in $\Gamma$; they can also perform weakenings, essentially
changing \verb+var+ field types into \verb+const+s. They must be
preventing from altering \verb+const+ and \verb+field+ slots.  To type
them, we'll need two auxiliary predicates:

\begin{itemize}
\item $\Gamma\vdash\update{L}{\T F}=[\sigma]$ checks whether
  variable/field $L$ is allowed to have its current type changed into
  $\T F$. If it is, it returns a substitution $[\sigma]$ over
  environments, so that $\Gamma[\sigma]$ is the typing environment in
  effect after the assignment has been performed.
\item $\T F_L \rhd \T F_R$ checks whether the content of slot of type
  $\T F_R$ can be stored in a slot of type $\T F_L$. It is, as we'll
  see, a subset of $:>$ the opposite of the subtyping relationship.
\end{itemize}

The former will prevent from changing the type of \verb+var+ fields,
and from changing the content of \verb+field+ or \verb+const+ fields:
there will be no rule allowing to derive $\Gamma\vdash\update{L}{\K\T
  E}=[\sigma]$; moreover, $\Gamma\vdash\update{L}{\V\T E}=[\ ]$ will
only be derivable from $\Gamma\vdash L\therefore\V\T E$, and will only
produce empty type substitutions $[\ ]$. Once update() has allowed an
assignment based on the field's former and new types, $\rhd$ checks
that the content put in the field is consistent with the new type, to
prevent such unsound assignments as {\tt v \#var number="abc"}.

The following rule allows to change the content of a \verb+var+ slot,
as long as its type isn't changed:

$$\frac
{\Gamma\vdash L \therefore \V\T E}
{\Gamma\vdash \update{L}{\V\T E} = [\ ]}
\eqlabel{UP-{\tt var}}
$$


\verb+currently+ slots can change the type of the value they contain,
but the slot type itself can also be changed, into a \verb+var+, a
\verb+const+ or even a \verb+field+.

$$\frac
{\Gamma\vdash v \therefore \C\T E}
{\Gamma\vdash \update{v}{\T F} = [v\leftarrow\T F]}
\eqlabel{UP-{\tt cur}}
$$

\verb+currently+ fields within tables pose an additional difficulty:
if the field's type changes, the type of the table containing it also
changes. Therefore, the table itself must also be stored in a
\verb+currently+ slot, etc.\ recursively until we reach a top-level
\verb+currently+ variable. The typing of assignments to those fields
is therefore defined recursively, with \eqlabel{UP-{\tt cur}} as a
base case, and \eqlabel{UP-{\tt cur}$[\ ]$} below as the inductive
rule:
%
$$\frac
{ 
  \begin{array}{c}
    \Gamma\vdash L \therefore \C[P:\C\T E; \phi]\\
    \Gamma\vdash \update{L}{\C[P:\T F; \phi]} = [\sigma]\\
  \end{array}
} {
  \Gamma\vdash\update{L[P]}{\T F} = [\sigma]
}
\eqlabel{UP-{\tt cur}$[\ ]$}
$$


As a usage example, let's consider an object \verb+x+ with a field
\verb+y+ currently containing a number, and updated to a string
variable. To make the proof tree terser, we'll use the following
definitions for \verb+x+'s former type $\T F_1$, its new type $\T
F_2$, and the typing environment before assignment $\Gamma$
respectively:
%
$$
  \begin{array}{l}
    \T F_1 = \C[\texttt{"y"}:\C\Tnumber]\\
    \T F_2 = \C[\texttt{"y"}:\V\Tstring]\\
    \Gamma = \{\texttt{x}\mapsto\T F_1\} \\
  \end{array}
$$
%
The soundness of environmnent substitution $[x\leftarrow\T F_2]$ is
computed by \eqlabel{UP-{\tt cur}} over {\tt x}; from there, it's
concluded by \eqlabel{UP-{\tt cur}$[\ ]$} that \verb+x["y"]+ can also
cause this substitution, because both \verb+x+ and \verb+x["y"]+ are
\verb+currently+ slots:

$$\frac
{
          \frac
              {\displaystyle \Gamma(\texttt{x}) = \T F_1}
              {\displaystyle \Gamma\vdash \texttt{x} \therefore \T F_1}
              \eqlabel{TR-$\therefore$}
    \qquad
    \frac
        {\displaystyle
          \frac
              {\displaystyle \Gamma(\texttt{x}) = \T F_1}
              {\displaystyle \Gamma\vdash \texttt{x} \therefore \T F_1}
              \eqlabel{TR-$\therefore$}
        }
        {\displaystyle \Gamma\vdash\update {\texttt{x}}{\T F_2} = [x\leftarrow \T F_2]}
        \eqlabel{UP-{\tt cur}}
}
{ \Gamma\vdash\update{\texttt{x["y"]}}{\T F_2} = [x\leftarrow\T F_2] }
\eqlabel{UP-{\tt cur}$[\ ]$}
$$

Because \verb+currently+ fields are only usable when they're inside
other \verb+currently+ fields all the way up to a variable, there's no
point having types such as $v \therefore \V[P:\C\T E; \phi]$: it
wouldn't allow anything more than $v \therefore \V[P:\V\T E; \phi]$.

~

$\T F_L \rhd \T F_R$ checks whether what's stored in a field has an
appropriate expression type. It also keeps track of linearity, forcing
to transform $\J\T E$ into $\C\T E$, and $\C\T E$ into $\F$. The
relation is expressed between two fields rather than a field and an
expression, to ease its recursive over tables (last rule below):

$$
\frac  {}  {\C\T E \rhd \J\T E}
%
\quad
%
\frac  {}  {\F \rhd \C\T E}
$$
$$
\frac  { }  {\V\T E \rhd \V\T E}
%
\qquad
%
\frac  { }  {\K\T E \rhd \K\T E}
%
\quad
%
$$
$$
\frac
{(\forall n\in[0...m]) \quad \T F_n^L \rhd \T F_n^R}
{
\C[(P_n:\T F_n^L)^{\forall n\in[1...m]}|\T F_0^L]
\rhd
\J[(P_n:\T F_n^R)^{\forall n\in[1...m]}|\T F_0^R]
}
$$
$$
\frac
    {\T F_L \rhd \T F_{R1} \quad \T F_{R1} :> \T F_{R2}}
    {\T F_L \rhd \T F_{R2}}
$$
$$\eqlabel{Accept}$$

Notice that although $\rhd$ is a subset of $:>$, it isn't an order
relayionship: it isn't idempotent (e.g. $\J\T E\not\rhd\J\T E$). By
using the composition with $:>$, we can choose to store a \verb+just+
field inside a table into either a \verb+currently+ or a \verb+var+
one; the former will allow to change the field's type, but any copy of
it can't be used (it will have to be further weakened into
\verb+field+); the latter will lock the type's content, but allows to
make and use further copies.

It is possible, but pointless, to put a \verb+currently+ field in a
\verb+var+ one: the outer \verb+var+ one will prevent from modifying
the inner one, thus making it strictly less usable than a \verb+var+
field (no type modification and no usable copy).

~

Equipped with these rules, we can now type assignments. But as an
intermediate step, we'll spell the simpler rule for the special case
where both left-hand side and right-hand side sequences have only one
element:

$$
\frac{
\begin{array}{c}
\Gamma\vdash E: \T E
\qquad
\T F \rhd \J\T E
\qquad
\Gamma\vdash \update{L}{\T F} = [\sigma]
\qquad
\Gamma[\sigma]\vdash \bar S:\T S
\end{array}
}{
\Gamma\vdash (L\ \#\T F=E; \bar S) : \T S
}
$$

To paraphrase, it must be possible (1) $E$ must be well typed; (2)
this type must be legal to store in a field of $L$'s new type $\T F$;
(3) it must be legal to substitute $L$'s former type with the new one
$\T F$; (4) the rest of the sequence $\bar S$ must be typable with
the type substitution $[\sigma]$ applied in $\Gamma$.

The actual rule, although more intimidating because it deals with
sequences of possibly different lengths, is not more
sophisticated. The two upper premices define a type $\bar\T E$
corresponding to $\T E$ above, with some \Nil-padding if needed to
match the right-hand-side's length. The lower one involving $\rhd$ and
update() corresponds to premices (2) and (3), applied to each (left,
right) pair; and the final premice chains all substitutions together
to type the following statements:

$$
\frac{
  \begin{array}{l}
    (\forall n\in [1...p])\ \Gamma\vdash E_n: \T E_n\quad
    (\forall n\in [p+1...m])\ \T E_n=\Nil\\
    (\forall n\in [1...m])\ 
    \T F_n \rhd \J\T E_n \textrm{ and }
    \Gamma\vdash \update{L_n}{\T F_n} = [\sigma_n]
  \end{array}
  %
  \Gamma[\sigma_1...\sigma_n] \vdash \bar S:\T S
}{
\Gamma\vdash ((L_n\ \#\T F_n)^{\forall n\in[1...m]}=(E_n)^{\forall n\in[1...p]}; \bar S) : \T S
}
$$
\vspace{-2em}
\begin{flushright}
  $\eqlabel{TR-Assign}$
\end{flushright}

\todo{Substitution conflicts within an assignment aren't handled,
  e.g. {\tt x \#[|currently nil] = \{ \}; x.a, x.b = "A", "B"}. The
  simplest solution is to mandate that substitutions have disjoint
  domains within an assignment.}

\paragraph{Functions}
Functions break linearity: they assign arguments to parameters, which
can create a second reference to a table. Moreover, due to Lua's
support for full closures, they capture variables defined outside of
them (these variables are called ``upvalues'' in Lua). Functions don't
use upvalues' content immediately, where the function is defined;
instead, they'll use them whenever they're called, and in between, the
content of a \verb+currently+ field or variable might have been
changed arbitrarily.

For this reason, when typing the function's body, we weaken every
 upvalue through $\rhd$, so that it doesn't rely on any variable
 having a field type of the form $\C\T E$. We elect to type parameters
 as \verb+var+ slots, thus allowing to change their content in the
 function's body. It would have been possible to forbid it by typing
 them with \verb+const+\footnote{I wonder whether typing parameters as
   {\tt currently} inside the function's body would have been
   admissible. It's not trivial: we must not allow to change a field
   in a table, so we must be sure there's no {\tt currently} table
   fields.}.

$$\frac{
  \Gamma_{\textrm{in}} \rhd \Gamma_{\textrm{out}}
  \quad
  \Gamma_{\textrm{in}}[\bar v\leftarrow\V\bar\T E_?]
  \vdash \bar S: \Return{\bar\T E_!}\\
}{
  \Gamma_{\textrm{out}}\vdash
  \Function{\overline{v\ \#\T E_?}}{\bar S}:
  \T E_?\rightarrow\bar\T E_!
}\eqlabel{TR-Function}$$

\noindent
(this rule generalizes $\rhd$ over environments, in the obvious way:
$\Gamma_1\rhd\Gamma_2$ iff $\Gamma_1$ has the same domain $D$ as
$\Gamma_2$, and $(\forall v \in D)\ \Gamma_1(v) \rhd \Gamma_2(v)$)


\paragraph{Function applications}
As we did for \eqlabel{TR-Assign}, let's demystify the function
application rule, by first giving the simplified version with one
parameter, one argument and one result: (1) what's applied must be a
function, (2) the argument must be well typed, and (3) this argument
type must be compatible with the parameter:

$$
\frac{
  \Gamma\vdash E_f: (\T E^?)\rightarrow(\T E^!)
  \qquad
  \Gamma\vdash E:\T E^a 
  \qquad
  \T E^a <: \T E^?
}{
\Gamma\vdash E_f(E^a)_E: (\T E^!)
}
$$

This rule is extended to multiple parameters / arguments / results
almost trivially; the only point to take care of is that the argument
types sequence might be padded with \Nil\ types, if it's shorter than
the parameter types sequence:

$$
\frac{
  \begin{array}{c}
    \Gamma\vdash E_f:
    (\T E^?_n)^{\forall n\in[1...m]}\rightarrow
    (\T E^!_n)^{\forall n\in[1...p]}\\
    (\forall n\in[1...q])\ \Gamma\vdash E^a_n:\T E^a_n
    \quad
    (\forall n\in[q+1...m])\ \T E^a_n = \Nil\\
    (\forall n\in[1...m])\ \T E^a_n <: \T E^?_n
  \end{array}
}{
\Gamma\vdash E_f((E^a_n)^{\forall n\in[1...q]})_E:
(\T E^!_n)^{\forall n\in[1...p]}
}
\quad
\eqlabel{TR-Apply-E}
$$

Finally, function applications in a statement context are typed by
discarding applying \eqlabel{TR-Apply-E}, then forgetting the results
and replacing them with the non-returning-statement type
$\varnothing_{\T S}$:

$$
\frac{
\Gamma\vdash E_f(\bar E)_E: \bar\T E
}{
\Gamma\vdash E_f(\bar E)_S: \varnothing_{\T S}
}
\quad
\eqlabel{TR-Apply-S}
$$

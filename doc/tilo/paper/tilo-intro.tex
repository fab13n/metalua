%-*-mode:latex; eval:(whizzytex-mode);-*-
%; whizzy-master tilo.tex

This article proposes a sound type system for the Lua programming
language. It's intended to be later combined with gradual typing and
partial type inference, so that users can blend statically and
dynamically typed program fragments, as best suits their development
needs.

Lua~\cite{lua} is a dynamic, imperative programming language, similar
in expressiveness to other modern languages such as Python, Ruby or
Javascript; compared to these, Lua specifically shines by its
embeddability, frugality in terms of hardware resources, tight
integration with C; its speed performances are also
noteworthy~\cite{perf-lua}, especially when run through a JIT
compiler~\cite{perf-luajit}. Finally, there is, by design, a proper
subset of the language which is widely acknowledged as
beginner-friendly~\cite{usability-ms}. Due to those qualities, Lua is
widely used in domains such as high-performance video games, embedded
devices or highly user-customizable systems.

Static typing is not always beginner-friendly; however, a well
designed, statically typed third party API is generally easier to use,
because many usage mistakes can be caught sooner, either when
compiling or immediately by a type-checking IDE. While introducing a
mandatory static type system in Lua would ruin several of the
language's key features, supporting optional types would significantly
improve the experience of many users.

We aim at offering such a fine-grained integration of static and
dynamic program fragments, by building upon the research on gradual
typing~\cite{gradual}. As a first step, we propose a type system,
inspired by theoretical studies of records and objects
typing~\cite{remy,sigma}, which accepts a significant proportion of
idiomatic Lua programs.

A preliminary, necessary step is the definition of a formal calculus
capturing the key semantic characteristics of Lua. Then we will
propose a type system for this calculus, which allows to check a
properly annotated program against illegal operations. We'll then hint
at ways to integrate gradual typing in this system, and partial
inference to lighten the amount of necessary annotations.

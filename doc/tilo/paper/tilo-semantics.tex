%-*-mode:latex; eval:(whizzytex-mode);-*-
%; whizzy-master tilo.tex

\section{Calculus}

This section defines a calculus, intended to capture Lua's defining
features. A compromise shall be found between faithfulness to Lua,
simplicity of the semantic rules, and ability to support a type
system.

\subsection{Notations}
\begin{itemize}

\item
  $\Sigma[x\leftarrow y]$, with $\Sigma$ a mathematical function,
  denotes the function which to $x$ associates $y$, and to all other
  values $x'$ associates $\Sigma(x')$ if defined.

\item
  $E[v\leftarrow E']$, with $E$ and $E'$ terms, and $v$ a variable
  possibly occurring in $E$, denotes the term $E$ in which all
  occurrences of $v$ are replaced with $E'$.

\item
  If $E$ denotes an element of a given kind, $\bar E$ denotes a
  sequence of such elements. A sequence has zero, one or several
  elements, which are ordered and not necessarily unique.

\item
  $\varnothing$ denotes an empty sequence. When it enhances
  understanding, it can be subscripted with the type of elements the
  sequence might have contained. For instance, an empty sequence of
  expressions might be denoted either $\varnothing$ or
  $\varnothing_E$.

\item
  $(x_n)^{\forall n\in[1...m]}$ denotes the sequence of all elements
  $x_n$ for successive values of $n$ $1, 2, ..., m$. Boundaries $1$
  and $m$ are inclusive.

\item
  Sequences are concatenated with a semicolon between parentheses:
  $(\bar E_1;\bar E_2)$. They can also be concatenated with single
  elements: $(E; \bar E)$.

\end{itemize}

\subsection{Terms definition}

In a first step, we'll introduce a calculus which captures Lua's
defining features, such as tables, functions, multi-value returns,
local variables. We will not deal with classic structures
(if/then/else statements, for loops etc.) which can be added later in
a rather straightforward way.

We also won't explicitly support features which can be easily
encoded. ``{\tt ...}'' trailing function arguments, method
invocations, globals, combined local variable declaration+assignment
are intentionally left out of the calculus.

We'll also consider it mandatory for function bodies to end with a
$\Return{\Nil}$, to save a specific rule about non-returning
functions (the Lua compiler performs this transformation for the same
reason).

Finally, we distinguish function applications as expressions $f(\bar
x)_E$ and as statements $f(\bar x)_S$: the latter discard whatever
value they might have returned. This distinction is trivial to do when
encoding a program into the calculus, and simplifies the calculus'
semantic rules.

$$
\begin{array}{rcll}
E &::=& L & \eqlabel{E-Left} \\
  &|&   P & \eqlabel{E-Primitive} \\
  &|&   E(\bar E)_E & \eqlabel{E-Apply} \\
  &|&   \Function{\bar v}{\bar S}  & \eqlabel{E-Function} \\
  &|&   \Table{ ([E^k_n]=E^v_n)^{\forall n\in[1...m]} } & \eqlabel{E-Table} \\
~\\
L &::=& v & \eqlabel{E-Variable} \\
  &|&   E[E] & \eqlabel{E-Index} \\
~\\
P &::=& \langle\textrm{string}\rangle & \eqlabel{E-String} \\
  &|&   \langle\textrm{number}\rangle & \eqlabel{E-Number} \\
  &|&   \texttt{true} & \eqlabel{E-True} \\
  &|&   \texttt{false} & \eqlabel{E-False} \\
  &|&   \texttt{nil} & \eqlabel{E-Nil} \\
  &|&   t & \eqlabel{E-TableRef} \\
  &|&   f & \eqlabel{E-ClosureRef} \\
~\\
S &::=& \Local{\bar v} & \eqlabel{S-Local} \\
  &|&   \bar L = \bar E & \eqlabel{S-Assign} \\
  &|&   E(\bar E)_S & \eqlabel{S-Apply} \\
  &|&   \Return \bar E & \eqlabel{S-Return} \\
\end{array}
$$

Expression elements are sorted into categories:
\begin{itemize}
\item $S$ denotes statements;
\item $E$ encompasses all expressions;
\item $P$ denotes primary expressions: expressions which evaluate to
  themselves, without side effect; we have $P\subset E$.
\item $L$ denotes left-hand-side values, i.e. values which can legally
  appear to the left of a $\bar L=\bar E$ assignment statement; we
  have $L\subset E$ and $L\cap P=\{\ \}$.
\end{itemize}

Table and function references $t$ and $f$ aren't part of Lua: they are
what tables and functions are evaluated to, so that the notion of
identity and mutation sharing in the calculus remain faithful to Lua.

\subsection{Semantics}

We'll now describe how calculus terms are evaluated. We'll do so in
big-step (``natural'') semantics, i.e. we give a proposition of the
form ``$\textrm{assumptions} \vdash \textrm{term to evaluate} \evalsto
\textrm{modified assumptions, fully evaluated term}$''. If no such
reduction statement can be proved, then the term to evaluate is
erroneous. This differs from small-step semantics, where one defines a
single reduction step, and the evaluation of a program is defined as
the repeated application of the reduction step until no more reduction
is possible.

More formally, the proposition $\Sigma_0, X_0\evalsto\Sigma_1, X_1$
reads ``element $X_0$, when evaluated under environment $\Sigma_0$,
reduces to $X_1$ and changes the environment into $\Sigma_1$''. To
clarify rules, this operator has been separated into $\evalstoE,
\evalstoEbar, \evalstoS, \evalstoSbar, \evalstoL, \evalstoLbar$
depending on the kind of terms on which it operates.

Rules will be presented the usual way in logic: as fraction bars, with
the premises over the bar, and the proved conclusion under
it. $a\ b\ c \over d$ means that proposition $d$ is proved if we can
provide a proof of propositions $a$, $b$ and $c$ (the ``premises''). A
rule with nothing above the line, such as ${}\over \ d\ $, is an
axiom: a proposition considered true with no need for any further
proof. A complete proof is therefore a tree of propositions, with the
goal proposition as the tree's root, and axioms as its leaves.

\paragraph{Identity of tables and closures}
In Lua, tables and functions have an identity, i.e. two structurally
equal tables are not equal, unless they're a shared reference to a
same local variable (or table element) holding the same value.  For
instance, in ``{\tt a=\{ \}; b=\{ \}}'', {\tt a} and {\tt b} are not
equal. Similarly, extensionally equal functions are not equal in the
Lua sense.

To reflect this in the calculus, we'll let ({\it E-Function}) and
({\it E-Table}) terms evaluate to a fresh reference $f$ or $t$ every
time they're evaluated; the association between the reference and its
definition will be kept in the environment store $\Sigma$. As a result,
equality works in the calculus as in Lua; for instance, in ``{\tt a=\{
  \}; b=a}'', {\tt a} and {\tt b} are indeed equal.

Moreover, in order to make mutable variables and first class functions
cohabit seamlessly, Lua defines up-values, i.e. local variables which
outlive their syntactical scope. To faithfully represent this in the
calculus, we'll perform an $\alpha$-renaming to fresh variable names
every time we encounter a ``$\Local{\bar v}$'' statement. This way,
all variables live forever in the calculus, although only up-values
might be referred to out of their scope. Put otherwise, the formal
calculus never performs garbage collection.

\paragraph{Program environment}
The program's environment, i.e. the assumptions under which a term is
evaluated, is represented by a triplet of functions $(\Sigma^L,
\Sigma^T, \Sigma^F)$. They keep track respectively of local variables'
content, tables' content and closures. Since in most cases the details
of the environment don't matter, the triplet is often shortened as
$\Sigma$.

Formally, the three functions have the following types:
\begin{itemize}
\item $\Sigma^L: v \mapsto P$ is a function from variable names 
  to the evaluated expressions they hold;
\item $\Sigma^T: (t\times P) \mapsto P$ is a function from
  table references and key expressions, to the value expressions held
  under this key in this table;
\item $\Sigma^F: f \mapsto (\bar v \times \bar S)$ is a function from
  function references $f$ to function definitions $\Function{\bar
    v}{\bar S}$.
\end{itemize}

The environment could also be used capture I/O operations, to describe
side effects other than variable value changes. This wouldn't
significantly change the calculus' semantic properties, though, and
hence will be left out.

\paragraph{Assignment to environment}
To properly define variable assignment, we'll need an operator
$\Leftarrow$ on environments, which denotes the update of what's bound
to variables and table fields in $\Sigma$. Indeed, an assignment
statement can update a mix of local variables and table contents, so
we need to describe the simultaneous update of $\Sigma^L$ and
$\Sigma^T$. The operator $(\Sigma^L, \Sigma^T) \Leftarrow (\bar L,
\bar P)$ is defined inductively as follows:
%
$$
\begin{array}{rcll}
%
(\Sigma^L,\Sigma^T) \Leftarrow (\varnothing_L,\bar P) &=& (\Sigma^L,\Sigma^T) &\eqlabel{A-$\varnothing_L$}\\ 
%
(\Sigma^L,\Sigma^T) \Leftarrow (\bar L,\varnothing_P)
&=&
(\Sigma^L,\Sigma^T) \Leftarrow (\bar L, \Nil)
&\eqlabel{A-$\varnothing_P$}\\
%
(\Sigma^L,\Sigma^T) \Leftarrow ((v;\bar L),(P;\bar P))
&=&
(\Sigma^L[v\leftarrow P], \Sigma^T) \Leftarrow (\bar L; \bar P)
&\eqlabel{A-Local}\\
%
(\Sigma^L,\Sigma^T) \Leftarrow (t[P_k], \bar L; P_v, \bar P)
&=&
(\Sigma^L,\Sigma^T[(t,P_k)\leftarrow P_v]) \Leftarrow (\bar L; \bar P)
&\eqlabel{A-Table}\\
%
\end{array}
$$

Intuitively, $\Leftarrow$ stores variable assignments in $\Sigma^L$
and table writings in $\Sigma^T$, thanks to the two last rules (the
two first ones are structural). It expects its third argument to be a
list of left-values, i.e. either variables or indexing of a primitive
term by another. As a notation facility, we'll allow to transparently
pass an extra $\Sigma^F$ argument to $\Leftarrow$. This lets use it
directly on complete environments $\Sigma$. Formally, the operator is
overloaded as follows:
%
$$(\Sigma^L,\Sigma^T,\Sigma^F) \Leftarrow (\bar L,\bar P) =
(\Sigma^L_\star,\Sigma^T_\star,\Sigma^F)
\textrm{ iff }
(\Sigma^L,\Sigma^T) \Leftarrow (\bar L, \bar P) =
(\Sigma^L_\star,\Sigma^T_\star)
$$.
%
\todo{ forbid $E_L[\Nil]=E_R$ }

\paragraph{Statements sequences evaluation}

$\varnothing_S$ denotes an empty sequence of statements. It is also the
result of a sequence which didn't return anything:

$$\frac{
}{
\Sigma, \varnothing_S \evalstoSbar \Sigma, \varnothing_S
}
\quad(\textit{ES-}\varnothing)$$

\noindent
If the first element of a sequence doesn't evaluate into a {\tt
  return}, then the result of the sequence is that of the following
statements (plus any side effect caused on $\Sigma$ by the first
statement):

$$\frac{
\Sigma, S^1 \evalstoS \Sigma_1, \varnothing
\quad
\Sigma_1, \bar S \evalstoSbar \Sigma_2, S^\star
}{
\Sigma, (S^1;\bar S) \evalstoSbar \Sigma_2, S^\star
}
\quad(\textit{ES-$\bar\varnothing$})$$

\noindent
However, if a statement evaluates to {\tt return},
the rest of the sequence isn't evaluated:

$$\frac{
\Sigma, S^1 \evalstoS \Sigma_1, \Return{\bar P}
}{
\Sigma, (S^1;\bar S) \evalstoSbar \Sigma_1, \Return{\bar P}
}
\quad(\textit{ES-}\overline{\tt return})$$

\paragraph{Statements}

\subparagraph{Return statements}
Return statements evaluate their returning values. When a function
body evaluates to $\Return{\bar P}$, it will be ``unwrapped'' back
into a $\bar P$ by ({\it EE-Apply}).
%
$$\frac{
\Sigma, \bar E \evalstoEbar \Sigma_1, \bar P
}{
\Sigma, \Return{\bar E} \evalstoS \Sigma_1, \Return{\bar P}
}
\quad(\textit{ES-Return})$$

\subparagraph{Local variables creation}
Local variable creations are immediately $\alpha$-renamed:
%
$$\frac{
\left(\Sigma^L\left[\bar w \leftarrow\overline\Nil\right],\Sigma^T,\Sigma^F\right), 
\bar S[\bar v \leftarrow \bar w]
\evalstoSbar \Sigma_1, S^\star
\quad
\bar w \textrm{\ free in }\Sigma^L
}{
(\Sigma^L,\Sigma^T,\Sigma^F), (\Local{\bar v}; \bar S)
\evalstoSbar \Sigma_1, S^\star
}\quad\eqlabel{ES-Local}$$

\noindent
The point is to handle upvalues (references to variables defined
outside of the function body) correctly. Consider for instance the
following program, featuring an up-value {\tt u}:
\begin{verbatim}
local u=1
local f = function(x)
  u=u+1
  return x+u
end
_ENV["a"], _ENV["b"] = f(1), f(2)
\end{verbatim}

\noindent
The first line {\tt local u} will be evaluated only once, and
therefore occurrences of {\tt u} in {\tt f} will all be
$\alpha$-renamed to the same fresh variable: they will indeed be
shared as expected. Conversely, in the following program:

\begin{verbatim}
local f = function(x)
  local u; u=1
  u=u+1
  return x+u
end
_ENV["a"], _ENV["b"] = f(1), f(2)
\end{verbatim}

The {\tt local u} statement will be evaluated twice, each time being
renamed in a different fresh variable, and no sharing can occur.

Evaluation of assignment contains a subtlety: the value-receiving
fields and variables, on the left of the ``$=$'' sign, must not be
fully evaluated. Instead, they need to be reduced to a ``left-normal''
form (variables or index to a table), which is done by a distinct
evaluation operator $\evalstoLbar$; once both left and right sides of
``$=$'' are evaluated, modifying the environment adequately is left to
the $\Leftarrow$ operator defined above:

$$
\frac{
\Sigma, \bar L \evalstoLbar \Sigma_1, \bar L^\star
\qquad
\Sigma_1, \bar E \evalstoEbar \Sigma_2, \bar P
}{
\Sigma_1, \bar L = \bar E
\evalstoS
(\Sigma_2\Leftarrow (\bar L^\star, \bar P)), \varnothing_S
}\quad(\textit{ES-Assign})
$$

We'll define $\evalstoLbar$ in terms of $\evalstoL$, which operates on
a single expression. Variables are considered fully evaluated when the
occur on the left of ``$=$'':

$$
\frac{
}{
\Sigma, v \evalstoL \Sigma, v
}\eqlabel{EL-v}$$

Indexed values have the table and its key evaluated, but the
field-content-accessing operation isn't performed:

$$
\frac{
\Sigma, E_T \evalstoE \Sigma_1, (P_T^1;\bar P_T)
\qquad
\Sigma_1, E_k \evalstoE \Sigma_2, (P_k^1;\bar P_k)
}{
\Sigma, E_T[E_k] \evalstoL \Sigma_2, P_T^1[P_k^1]
}\eqlabel{EL-Index}$$

With this we can easily define $\evalstoLbar$, which chains $\evalstoL$
operations by stringing their environment modifications together:

$$
\frac{
\Sigma, L_1 \evalstoL \Sigma_1, L_1^\star
\qquad
\Sigma_1, \bar L \evalstoLbar \Sigma_2, \bar L^\star
}{
\Sigma, (L_1;\bar L) \evalstoLbar \Sigma_2, (L_1^\star; \bar L^\star)
}\eqlabel{EL-$\bar L$}
\qquad
\frac{}{\Sigma, \varnothing_L \evalstoLbar \Sigma, \varnothing_L}
\eqlabel{EL-$\varnothing$}
$$

\subparagraph{Function applications in statment contexts}
Function application in a statement context is pretty similar to
function application in an expression context, except that any
returned result is thrown out.  The actual
$\beta$-reduction is therefore delegated to $\eqlabel{EE-Apply}$,
defined later:

$$ \frac{
  \Sigma, E_f(\bar E)_E \evalstoE \Sigma_1, \bar P\\
  }
  {\Sigma, E_f(\bar E)_S \evalstoS \Sigma_1, \varnothing_S}
  \quad
  (\textit{ES-Apply})$$

\paragraph{Expression sequences}
In Lua, when a single expression evaluates into several results, only
the first result is kept, except for the last one which is entirely
appended to the resulting multi-value.  For instance, if we consider
{\tt f = function(a,b,c) return a,b,c end}, the sequence {\tt
  f(10,11,12), f(20,21,22), f(30,31,32)} will evaluate to {\tt 10, 20,
  30, 31, 32}.

The rule below chains expression evaluations by stringing their
environment modifications together, and discards extraneous returned values:

$$\frac{
  (\forall n \in [1...m])\ 
  \Sigma_{n-1}, E_n \evalstoE \Sigma_n, (P_n^1;\bar P_n)
} {
  \Sigma_0, (E_n)^{\forall n\in [1...m]} \evalstoEbar \Sigma_m,
  ((P_n^1)^{\forall n\in[1...m]}; \bar P_m)
}
\quad\eqlabel{EE-Sequence}$$

\paragraph{Expressions}

\subparagraph{Variables and primitives}
Variables are replaced by their content from the store; primitives are
their own evaluation:

$$ \frac{
\Sigma^L(v) = P
}{
(\Sigma^L,\Sigma^T,\Sigma^F), v \evalstoE P
}\quad\eqlabel{EE-v}
\qquad
\frac{}{
\Sigma, P \evalstoE \Sigma, P
}\quad\eqlabel{EE-Primitive}$$

\subparagraph{Function application}
For function applications, function definitions are retrieved from
$\Sigma^F$. We define the evaluation by transforming the function
parameters into local variables, assigned to the arguments' values.
Notice that the arguments are evaluated before their assignment,
although \eqlabel{EE-Assign} would have evaluated them anyway. The
reason is, the arguments need to be evaluated outside of the
function's scope; otherwise, capture problems could occur (think for
instance of {\tt f=function(x) return x end; local x; f(x)}):

$$ \frac{
  \begin{array}{c}
    \Sigma, E_f \evalstoE (\Sigma_1^L,\Sigma_1^T,\Sigma_1^F), f\\
    \Sigma^F_1(f) = \Function{\bar v}{\bar S}\\
    (\Sigma_1^L,\Sigma_1^T,\Sigma_1^F), \bar E \evalstoEbar
    \Sigma_2, \bar P\\
    \Sigma_2, (\Local{\bar v}; \bar v = \bar P; \bar S)
    \evalstoSbar
    \Sigma_3, \Return{\bar P}
  \end{array}
  }
  {\Sigma, E_f(\bar E)_E \evalstoE \Sigma_3, \bar P}
  \quad
  (\textit{EE-Apply})$$

\subparagraph{Function creations}
As seen in \eqlabel{EE-Apply}, function definitions are retrieved from
$\Sigma^F$. They're stored in it, under a fresh name $f$, when the
{\tt function ... end} expression is found:

$$\frac{
  E_f = \Function{\bar v}{\bar S}
  \qquad
  f \textrm{ free in } \Sigma^F
}{
  (\Sigma^L,\Sigma^T,\Sigma^F), E_f
  \evalstoE 
  (\Sigma^L,\Sigma^T,\Sigma^F[f\leftarrow E_f]), f
}\quad\eqlabel{EE-Function}$$


\subparagraph{Literal tables}
To evaluate a literal table, we evaluate every key and value in order,
chaining environment modifications, then store the (key, value)
evaluated pairs in $\Sigma^T$:

$$\frac{
  \begin{array}{c}
    t \textrm{ free in } \Sigma_{2m}^T\\
    (\forall n\in[1...m])\ 
    \left\{
    \begin{array}{l}
    \Sigma_{2n-2}, E_k^n \evalsto \Sigma_{2n-1}, (P_k^n; \bar P_k^n)\\
    \Sigma_{2n-1}, E_v^n \evalsto \Sigma_{2n\phantom{-1}}, (P_v^n;
    \bar P_v^n)\\
    \end{array}\right.\\
    \Sigma^T_\star = \Sigma_{2m}^T
    [(t, P_k^n)\leftarrow P_v^n]^
    {\forall n\in[1...m]}
  \end{array}
}{
\Sigma_0, ([E_k^n]=E_v^n)^{\forall n\in[1...m]}\\
\evalstoE
(\Sigma_{2m}^L,\Sigma^T_\star,\Sigma_{2m}^F), t\\
}\quad\eqlabel{EE-Table}$$

\subparagraph{Accessing table contents}
When a value is indexed, it must evaluate to a table reference; then
the value associated with the corresponding key is retrieved from the
store $\Sigma^T$:

$$\frac{
\Sigma,E_T \evalsto \Sigma_1, t
\quad
\Sigma_1, E_k \evalsto \Sigma_2, (P_k^1;\bar P_k)
\quad
\Sigma_2^T(t, P_k^1) = P_v
} {
\Sigma, E_T[E_k] \evalsto \Sigma_2, P_v
}\quad\eqlabel{EE-Index}$$

In Lua, a key which has never been set in a table is associated with
value \Nil. To reflect this, an evaluation must start with all
table/key pairs associated with $\Nil$:

$$(\forall t)(\forall P)\ \Sigma^T(t, P) = \Nil$$

\subsection{How evaluation can fail}

The operational semantics above defines the result of programs, as
long as they:
\begin{itemize}
\item don't get stuck in infinite recursion;
\item don't try to index a non-table (cf. premises of
  $\eqlabel{EE-Index}$, which is the only evaluation rule applying to
  terms of the form $E[E]$, and requires the indexed object to be a
  reference to a table);
\item don't try to apply a non-function (cf. premises of
  $\eqlabel{EE-Apply}$, which is the only evaluation rule applying to
  terms of the form $E(\bar E)$, and requires the applied object to be
  a reference to a function;);
\item always return a value from a function, i.e. all function bodies,
  when parameters are substituted with arguments, evaluate to a
  $\Return{\bar P}$ statement value. It's easily proved that by
  appending a $\Return{\Nil}$ at the end of the function's body, a
  function body evaluating to $\varnothing_S$ will evaluate to
  $\Return{\Nil}$ instead.
\end{itemize}

The last point is very easily addressed, and the first one is well
known as undecidable; the most reasonable definition of static
correctness, for a program, is to provably perform no indexing of a
non-table value, and no function call on a non-function. A sound type
system for the present calculus will provide formal proofs that a
given term cannot involve such incorrect sub-terms. The design of such
a type system is the subject of the next section.

%-*-mode:latex; eval:(whizzytex-mode);-*-

\section{Terms annotation}

In this section, we propose a way to annotate Lua terms with
types. We'll talk about concrete syntax matters here, not only formal
considerations.

First let's define the fully annotated calculus, holding all the
typing annotations potentially needed to typecheck a program. We'll
later discuss the possibility to leave some annotations missing, and
how they should be interpreted. We'll also leave out facilities such
as type alias definitions for now: the goal here is to determine how
much annotation is needed to check a term.

$$
\begin{array}{rcll}
E &::=& L & \eqlabel{E-Left} \\
  &|&   P & \eqlabel{E-Primitive} \\
  &|&   E(\bar E)_E & \eqlabel{E-Apply} \\
  &|&   \Function{\bar A}{\bar S}  & \eqlabel{E-Function} \\
  &|&   \Table{ \compseq{[E^k_n]=E^v_n}{n\in[1...n]} } & \eqlabel{E-Table} \\
L &::=& v\ \sharp \T E & \eqlabel{E-Variable} \\
  &|&   E[E] & \eqlabel{E-Index} \\
P &::=& \langle\textrm{string}\rangle & \eqlabel{E-String} \\
  &|&   \langle\textrm{number}\rangle & \eqlabel{E-Number} \\
  &|&   \texttt{true} & \eqlabel{E-True} \\
  &|&   \texttt{false} & \eqlabel{E-False} \\
  &|&   \texttt{nil} & \eqlabel{E-Nil} \\
  &|&   t & \eqlabel{E-TableRef} \\
  &|&   f & \eqlabel{E-ClosureRef} \\
S &::=& \Local{\bar v} & \eqlabel{S-Local} \\
  &|&   \bar L = \bar E & \eqlabel{S-Assign} \\
  &|&   E(\bar E)_S & \eqlabel{S-Apply} \\
  &|&   \Return \bar E & \eqlabel{S-Return} \\
  &|&   \sharp \Return{\bar\T E}\\
A &::=& v\ |\  v\ \sharp \T E\\
\T E &::=& [\T F; \overline{P:\T F}]\\
  &|&   v\\
  &|&   \star_{\T E}\\
  &|&   \T E[P]\\
  &|&   \bar\T E \rightarrow \bar\T E\\
\bar\T E &::=& \star_{\bar\T E}\ |\ v_{\bar\T E}\ |\ (\overline{ \T E}) \\
\T F &::=& \Tcurrently{\T E}\\
  &|&   \Tvar{\T E}\\
  &|&   \Tconst{\T E}\\
  &|&   \Tfield\\
\end{array}
$$

With this annotation system:
\begin{itemize}
\item function parameters are explicitly typed; The result is
  inferred, but can be forced with a statement annotation;
\item locals aren't typed at decalaration (they contain \Nil\ anyway)),
  but variable assignmnents are annotated. This allows to specify
  variable type changes when allowed by the type system;
\item tables are not annotated, but as soon as they're assigned in a
  variable, this assignement operation is annotated;
\item table field updates. Overriding \Tvar{}fields is OK as long as
  the type doesn't change; for \Tcurrently{}fields, we need to guess
  the type. It probably doesn't matter that much, since those types
  are easily overridden; \todo{a covariant subtyping rule
    might be missing here, ot go from e.g. {\tt currently positive} to
    {\tt var number}}
\item function applications are typed are deduced from the function's
  type;
\item of course $P$ values are typed;
\item table field accesses are only typed if the table is typed;
\item return statements are inferred.
\end{itemize}


Unannotated function parameters will be treated as dynamic types;
we'll try to guess a static type for unannotated local vars, though:
if we don't, all function calls using them will become dynamic, and
little actual type-checking will be able to take place.

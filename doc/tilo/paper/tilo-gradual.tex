%-*-mode:latex; eval:(whizzytex-mode);-*-
%; whizzy-master lucal.tex

\section{Gradual typing extension}

This section extends the type system with the dynamic type
$\star$. Its integration with the static type system is done through
gradual typing, in a way similar to several such integrations. The
concept of gradual typing is introduced in \cite{XXX}; its sound
integration with subtyping among dynamic types is studied in
\cite{XXX}. Several papers explore the ability to integrate it with
type inference \cite{xxx}\cite{xxx}.

The fundamental principle is that a dynamic type $\star$ is
introduced; terms typed dynamically are ``trusted'' by the
type-checker, i.e. they can be used in any typing context. The formal
definition of this is based on a compatibility equivalence
relationship $\sim$, which states that two types can be made equal
through a substitution of $\star$ occurrences in both of them. For
instance, we have $\Tnumber\sim\star$, $\Tnumber\sim\Tnumber$ and
$(\Tnumber\rightarrow\star)\sim(\star\rightarrow\Nil)$, but
$\Tnumber\nsim\Tstring$, $\Tnumber\nsim[\Tconst \star]$ and
$(\Tnumber\rightarrow\star)\nsim(\Tstring\rightarrow\Nil)$.

The compatibility relationship is orthogonal with subtyping; however,
by applying the subtyping relationship to the equivalence classes
defined by compatibility, one gets the partial order $\lesssim$: 

$$
\frac{
\T T_1 <: \T T_2
\quad
\T T_2 \sim \T T_3
}{
\T T_1 \lesssim \T T_3
}
$$

We'll introduce two dynamic types: one for expressions $\star_{\T E}$,
and one for expression types $\star_{\bar{\T E}}$. Some rules will
have to be rewritten to use $\lesssim$ instead of $<:$, some will be
added to support $\star$ types,
